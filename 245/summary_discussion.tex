\section{Zusammenfassung und Diskussion}

In Versuch 245 untersuchten wir in verschiedenen Aufbauten die elektromagnetische Induktion. Mathematisch fundiert durch das Induktionsgesetz, besagt dieses, dass durch die zeitliche Änderung des magnetischen Flusses ein elektrischer Strom erzeugt wird. Zu Erzeugung des Magnetfeldes verwendeten wir eine Helmholtz-Spule, dies ist ein Aufbau aus zwei in Reihe geschalteten Spulen, in dessen Zentrum ein nahezu homogenes Magnetfeld erzeugt werden kann. In eben diesem Zentrum war eine kleinere, drehbare Magnetspule platziert, in welcher der Strom durch das Feld der Helmholtz-Spule induziert wurde. Das Induktionsgesetz gibt verschiedene Größen an, von welchen die induzierte Spannung abhängt.

Im ersten Versuchsteil untersuchten wir die Auswirkungen der Drehfrequenz der Spule, sowie der magnetischen Flussdichte des äußeren Magnetfelds auf die Induktionsspannung. Aus den aufgenommenen Daten bei der Untersuchung der Frequenzabhängigkeit berechneten wir die magnetische Flussdichte im Zentrum der Helmholtz-Spule zu
\begin{align*}
  (3.63 \pm 0.08) \cdot 10^{-3}\, \si{T}.
\end{align*}
Der theoretische Wert für diese Größe lässt sich aus den Eigenschaften der Spule zu $(3.02 \pm 0.08) \cdot 10^{-3}\, \si{T}$ berechnen. Die Abweichung könnte zum einen durch weitere äußere Felder im Versuchsraum, welche neben dem Feld der Helmholtz-Spule auf die Induktionsspule wirkten, verursacht sein. Idealisierte Annahmen in der Theorie, welche in der Realität nicht gegeben sind, haben ebenfalls Auswirkungen auf das Ergebnis, genauso wie Ungenauigkeiten in der Elektronik des Versuchsaufbaus.

Im zweiten Versuchsteil analysierten wir die Induktionsspannung bei einem periodisch veränderlichen äußeren Feld, dazu betrieben wir die Helmholtz-Spule mit einer Wechselspannung. Wir betrachteten zunächst die induzierte Spannung für verschiedenen Winkelstellungen der Induktionsspule, dann für verschiedene Frequenzen der Wechselspannung an der Helmholtz-Spule. Aus den hierbei aufgenommenen Daten berechneten wir die Induktivität der Helmholtz-Spule zu
\begin{align*}
  L = (0.0605 \pm 0.0014)\, \si{H}.
\end{align*}
Der dritte Versuchsteil befasste sich mit dem Magnetfeld der Erde. Wenn wir die Induktionsspule entlang der Feldlinien, also in Nord-Süd-Richtung ausrichteten und diese in Rotation versetzten, wurde durch das Erdmagnetfeld in der Spule eine Spannung induziert. Durch Anwendung des Induktionsgesetzes war es uns möglich, aus der gemessenen Induktionsspannung die magnetische Flussdichte des Erdmagnetfeldes zu berechnen. Hier kamen wir auf einen Wert von
\begin{align*}
  B_{E} = (47.7 \pm 0.7)\unit{\micro\tesla}.
\end{align*}
Auf der Erdoberfläche treffen die Feldlinien des Erdmagnetfeldes im sogenannten Inklinationswinkel auf. Das untersuchte Magnetfeld lässt sich somit in eine Vertikal- und Horizontalkomponente zerlegen. Durch zuschalten der Helmholtz-Spule konnten wir die Vertikalkomponente kompensieren und somit \glqq{}isoliert\grqq{} die magnetische Flussdichte der Horizontalkomponente nach demselben Schema wie zuvor aus der Induktionsspannung berechnen. Mithilft trigonometrischer Funktionen ließ sich damit der Inklinationswinkel am Versuchsstandort (Heidelberg) zu
\begin{align}
  \alpha = (72.1 \pm 2.7) \si{\degree}
\end{align}
berechnen.

Mithilfe des \textit{IGRF Declination Calculator} des Helmholtz Centre Potsdam ist es möglich, Referenzdaten für des Erdmagnetfeldes an verschiedenen Standorten abzurufen. Für Heidelberg, bei einer Höhe von $114\si{\meter}$ bei den Koordinaten $49\si{\degree} 25'$N $8 \si{\degree} 42'$O ist eine gesamte magnetische Flussdichte von $48939.3 \si{\nano\tesla}$ angegeben. Dieser Wert liegt im $2\sigma$-Bereich unseres errechneten Wertes. Für den Inklinationswinkel in Heidelberg gibt die Referenz einen Wert von $65.25\si{\degree}$ an. Dieser Wert weicht etwas stärker von dem von uns berechneten Wert ab, liegt aber noch im $3\sigma$-Bereich um diesen. Fehlerursachen könnten hier beispielsweise wieder die Auswirkungen überlagernder äußerer Felder im Versuchsaufbau sein. Ungenauigkeiten bei der Ausrichtung der Spule mit dem Kompass könnten hier auch eine Rolle spielen, sowie die Einstellung des Kompensationsfeldes nach Augenmaß.

