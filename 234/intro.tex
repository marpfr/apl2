In Versuchs 234 betrachten wir verschiedene Arten von Lichtquellen und deren Spektren. Hierbei unterscheiden wir speziell zwischen Temperaturstrahlern und Nichttemperaturstrahlern, welche in ihrer Art und Weise, Licht bzw. Strahlung auszusenden auf zwei verschiedenen physikalischen Phänomenen basieren.

\subsection{Physikalische Grundlagen}

Zur Erklärung von \textbf{Temperaturstrahlern} betrachten wir zunächst einen idelisierten schwarzen Körper, also einen Körper mit dem Emissionsvermögen $\varepsilon = 1$. Sobald die Temperatur $T$ dieses, und allgemein jedes Körpers, größer $0 \si{\kelvin}$ ist, sendet dieser elektromagnetische Strahlung aus, deren Intensität dem Planck'schen Strahlungsgesetz
\begin{align}
  M_{\lambda} (\lambda, T) \dd{A} \dd{\lambda} = \frac{2 \pi h c^2}{\lambda^5} \frac{1}{\e{\frac{hc}{\lambda k T}} - 1} \dd{A} \dd{\lambda}
\end{align}
folgt. $M_{\lambda}$ steht hierbei für die Strahlungsleistung, welche vom Flächenelement $\dd{A}$ im Wellenlängenbereich $\qty[\lambda, \lambda + \dd{\lambda}]$ abgestrahlt wird. Die Intensitätsverteilung besitzt ein Maximum bei einer bestimmten Wellenlänge, welche nach dem Wie'schen Verschiebungsgesetz
\begin{align}
  \lambda_{\max} = \frac{2897.8 \si{\micro\meter\kelvin}}{T}
\end{align}
von der Temperatur des Körpers abhängt.

Durch Betrachtung der Farbe des ausgestrahlten Lichts lassen sich dich Lichtquellen charakterisieren. Ein Körper erscheint bei einer Temperatur von $0\si{\kelvin}$ absolut schwarz. Darauf folgt bei leichter Erwärmung die Emission von zunächst rotem Licht. Bei einer weiteren Erwärmung wird zusätzlich grünes licht emittiert, welches sich mit den Rotanteilen zu orange bis gelblich erscheinendem Licht vermischt. Weißlich wirkendes Licht erhält man, wenn sich bei einer Temperatur von etwa $5500 \si{\kelvin}$ alle Wellenlängen des sichtbaren Bereichs mit etwa gleicher Intensität emittiert werden. Mit weiterer Erwärmung des Körpers kommen vermehrt Blauanteile hinzu, welche das Licht zunächst hellblau, dann blau bis violett erscheinen lässt.

\subsection{Versuchsdurchführung}