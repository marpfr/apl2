\section{Zusammenfassung und Diskussion}

In Versuch 234 beschäftigen wir uns mit der Untersuchung der Spektren verschiedener Lichtquellen. Hier unterscheiden wir zwischen Temperaturstrahlern und Nichttemperaturstrahlern. Temperaturstrahler, wie beispielsweise eine Glühlampe, basieren auf dem Phänomen, dass jeder Körper, dessen Temperatur größer als $0\si{\kelvin}$ ist, elektromagnetische Strahlung abstrahlt, deren Intensitätsverteilung abhängig von der Wellenlänge dem Planck'schen Strahlungsgesetz folgt. Bei den Nichttemperaturstrahlern ist die Erzeugung von Licht entweder auf Anregung von Atomzuständen, zum Beispiel bei einer Natriumdampflampe, oder der Rekombination von Elektron-Loch-Paaren in Halbleitern, also LEDs, zurückzuführen.

Im ersten Versuchsteil untersuchten wir das Spektrum des wohl bekanntesten Temperaturstrahlers, der Sonne. Mit einem Gitterspektroskop nahmen wir das Spektrum des Tageslichts einmal direkt und einmal durch ein Fenster auf. Wir konnten beobachten, dass es sich um ein kontinuierliches Spektrum handelt, wie es von Temperaturstrahlern zu erwarten ist. 

\begin{figure}[H]
  \centering
  \includegraphics[width=.9\textwidth]{files/plots/himmel_m_o_g.png}
  \caption{Tageslichtspektrum direkt (blau) und durch Fensterglas (orange).}
  \label{fig:himmel_m_o_g_zsmf}
\end{figure}


Im Vergleich des Spektrums durch das Fenster mit dem, das direkt aufgezeichnet wurde, konnten wir beobachten, dass durch die Scheibe über den gesamten Wellenlängenbereich hinweg die Intensität des Lichts durch die Absorption der Glasscheibe abgeschwächt wird. Bei genauerer Betrachtung der Absorption konnten wir sehen, dass diese im Bereich von Wellenlängen unter $400\si{\nano\meter}$, also im nicht-sichtbaren UV-Bereich, am höchsten und im Bereich des sichtbaren Lichts am niedrigsten ist.

Die vielen im Spektrum deutlich sichtbaren lokalen Minima sind durch die Absorption von Licht bestimmter Wellenlängen in den Atmosphärenschichten der Sonne und der Erde zu erklären. Diese Absorptionslinien, genannt \glqq{}Fraunhoferlinien\grqq{} sind speziellen Wellenlängen zugeordnet, welche wir mit den Positionen der Linien in unsren aufgezeichneten Spektrum verglichen. Die Werte sind, mit der jeweiligen Abweichung vom Literaturwert in \tabref{tab:fraunhofer_vergleich_zsmf} zusammengefasst. Als Fehler für die beobachteten Wellenlängen sind wir jeweils von $\pm 1\si{\nano\meter}$ ausgegangen.

\begin{table}[h]
  \centering
  \caption{Vergleich der erwarteten und gemessenen Wellenlängen der Fraunhofer- und Balmerlinien}
  \vspace*{0.5em}
  \begin{tabular}{c|c|c|c}
      \hline
      Linie & Literaturwert [nm] & Abgelesener Wert [nm] & Abweichung [$\sigma$] \\
      \hline
      K  & 393.4 & 393.0 & 0.4 \\
      H  & 396.8 & 396.1 & 0.7 \\
      G  & 430.8 & 429.8 & 1.0 \\
      F  & 486.1 & 485.2 & 0.91 \\
      b1 & 518.4 & 516.7 & 1.7 \\
      E  & 527.0 & 526.2 & 0.8 \\
      D3 & 587.6 & 588.4 & 0.8 \\
      D2 & 589.0 & 589.0 & 0.0 \\
      D1 & 589.6 & 589.7 & 0.11 \\
      C  & 656.3 & 655.0 & 1.3 \\
      B  & 686.7 & 686.7 & 0.0 \\
      A  & 759.4 & 759.4 & 0.0 \\
      \hline\hline
      $\mathrm{H}_{\alpha}$ & 656.3 & 655.0 & 1.3\\
      $\mathrm{H}_{\beta}$ & 486.1 & 485.2 & 0.91\\
      $\mathrm{H}_{\gamma}$ & 434.0 & 433.4 & 0.61\\
      $\mathrm{H}_{\delta}$ & 410.1 & 409.5 & 0.61\\
      \hline
  \end{tabular}
  \label{tab:fraunhofer_vergleich_zsmf}
\end{table}

Unter anderem als lokale Minima im Sonnenlichtspektrum zu beobachten sind die Linien der Balmerserie, welche auf Anregungen in Wasserstoffatom zurückzuführen sind. Die beobachteten Positionen der $H_{\alpha}-, H_{\beta}-, H_{\gamma}-, H_{\delta}-$Linien der Balmerserie verglichen wir ebenfalls mit den Literaturwerten, zusammengefasst in derselben Tabelle.

Insgesamt ist die Abweichung der beobachteten Absorptionslinien von den Literaturwerten sehr gering. Die dennoch sichtbaren Unterschieden sind vermutlich zu Großteilen auf das wolkige Wetter am Versuchstag, sowie Störungen und Reflexionen durch die umliegenden Gebäude zurückzuführen.

Im darauf folgenden Versuchsteil betrachteten wir qualitativ die Spektren verschiedener Lichtquellen. Hierunter untersuchten wir das Licht verschiedenfarbiger LEDs, eines Lasers und einer Energiesparlampe als Beispiele für Nichttemperaturstrahler, sowie das einer Glühlampe als ein klassisches Beispiel für einen Temperaturstrahler. Wie in der Theorie beschrieben, konnten wir beobachten, dass die LEDs, der Laser und die Energiesparlampe diskrete Spektren aufweisen. Dabei basiert die Erzeugung von weißem Licht bei den weißen LEDs und der Energiesparlampe auf verschiedenen Methoden. Die Glühlampe wies ein kontinuierliches Spektrum auf, von welchem große Teile außerhalb des sichtbaren Wellenlängenbereichs lagen. Dies zeigte auch, dass diese im Vergleich zu den anderen Lichtquellen viel weniger energieeffizient ist.

Im abschließenden großen Versuchsblock setzen wir uns mit dem Spektrum einer Natriumdampflampe auseinander. Als Beispiel für eine Gasentladungslampe weist diese ein diskretes Spektrum auf, welches durch eine Vielzahl an Spektrallinien über den gesamten beobachteten Wellenlängen charakterisiert ist. Wir betrachteten zunächst das Spektrum in einem Wellenlängenbereich von $350$ bis $550\si{\nano\meter}$. Hier sind Spektrallinien mit geringer Intensität zu beobachten. Wir notierten die Wellenlängen aller gut sichtbarer Linien für den späteren Vergleich. Im Bereich zwischen $590$ und $600\si{\nano\meter}$ befindet sich im Natriumspektrum die markante D-Linie. Auch die Wellenlänge dieser Linie selbst, sowie die Wellenlängen einiger umliegender Linien notierten wir uns. Zuletzt zeichneten wir noch die Wellenlängen einiger markanter Linien im oberen Wellenlängenbereich von $650$ bis $850\si{\nano\meter}$ auf.

Zum Vergleich der beobachteten mit den theoretisch vorhersagten Positionen der Spektrallinien betrachteten wir dann die Übergänge $md \to 3p$ der 1. Nebenserie, die Übergänge $ms \to 3p$ der 2. Nebenserie, sowie die $mp \to 3s$ der Hauptserie. Die Energie $E_{3p}$, welche wir für die Berechnungen der Wellenlängen der Übergänge zum $3p$-Zustand benötigen, bestimmten wir auf einen Wert von
\begin{align}
  E_{3p} = -3.0279 \pm 0.0019\si{\electronvolt}.
\end{align}

Mit diesem bestimmten wir anhand der Formel $\lambda_m \approx \flatfrac{hc}{\qty(\frac{E_{Ry}}{m^2} - E_{3p})}$ für die Quantenzahlen $m=3,\dots,12$ die erwarteten Wellenlängen der Spektrallinien. Diese sind in \tabref{tab:wellenlaengen_1ns_zsmf} zusammen mit den von uns beobachteten Wellenlängen, welche wir diesen zuordnen konnten, aufgelistet.

\begin{table}[H]
  \centering
  \caption{Vergleich der berechneten und gemessenen Wellenlängen der 1. Nebenserie $md \to 3p$.}
  \vspace*{0.5em}
  \begin{tabular}{c c c c}
      \hline
      $m$ & $\lambda_{\text{theo.}}$ [nm] & $\lambda_{\text{beob.}}$ [nm] & Abweichung \\
      \hline
      3  & $817.7 \pm 1.000$ & $817.7 \pm 1$ & 0.01$\sigma$ \\
      4  & $569.4 \pm 0.5$ & $567.1 \pm 1$ & 2.03$\sigma$ \\
      5  & $499.2 \pm 0.4$ & $496.9 \pm 2$ & 1.13$\sigma$ \\
      6  & $467.9 \pm 0.4$ & $469.2 \pm 1$ & 1.28$\sigma$ \\
      7  & $450.8 \pm 0.4$ & $450.0 \pm 1$ & 0.77$\sigma$ \\
      8  & $440.4 \pm 0.3$ & $438.1 \pm 1$ & 2.20$\sigma$ \\
      9  & $433.5 \pm 0.3$ & $432.6 \pm 1$ & 0.88$\sigma$ \\
      10 & $428.7 \pm 0.3$ & $429.2 \pm 1$ & 0.46$\sigma$ \\
      11 & $425.3 \pm 0.3$ & $426.0 \pm 1$ & 0.72$\sigma$ \\
      12 & $422.7 \pm 0.3$ & $423.1 \pm 1$ & 0.44$\sigma$ \\
      \hline
  \end{tabular}
  \label{tab:wellenlaengen_1ns_zsmf}
\end{table}

Wir können sehen, dass die Abweichung bei den meisten Linien unter einem $\sigma$ liegt. Bei den Vergleichen mit einer signifikanteren Abweichung kann es durchaus sein, dass den theoretisch vorhersagten Linien falsche beobachtete Linien zugeordnet wurden.

Analog gingen wir für die 2. Nebenserie vor. Für die Berechnung der Wellenlängen benötigen wir hierbei zusätzlich den Korrekturterm $\Delta_s$. Um diesen zu bestimmen, berechneten wir zunächst die Energie des Zustandes $3s$ zu
\begin{align}
  E_{3s} = (-5.127 \pm 0.005)\si{\electronvolt},
\end{align}
und daraus wiederum den Korrekturterm
\begin{align}
  \Delta_s = 1.3711 \pm 0.0007.
\end{align}

Dieser ging dann mit in die Formel $\lambda_m \approx \flatfrac{hc}{\qty(\frac{E_{Ry}}{(m - \Delta_s)^2} - E_{3p})}$ mit ein, um die Wellenlängen für die Quantenzahlen $m=4,\dots,9$ zu bestimmen. Die Resultate sind in \tabref{tab:wellenlaengen_2ns_zsmf} noch einmal wiedergegeben. Da diese außerhalb des untersuchten Bereichs liegt, konnten wir der Spektrallinie für den Übergang $3s\to3p$ keine beobachtete Wellenlänge zuordnen. Bei allen weiteren sehen wir erneut, dass die Abweichung zwischen den jeweils zugeordneten Linien weitestgehend gering ausfällt. Lediglich die Linie der Quantenzahl $m=7$ weicht etwas stärker ab, was möglicherweise auf eine falsche Zuordnung zurückzuführen ist.

\begin{table}[H]
  \centering
  \caption{Vergleich der berechneten und gemessenen Wellenlängen der 2. Nebenserie $ms \to 3p$.}
  \vspace*{0.5em}
  \begin{tabular}{c c c c}
      \hline
      $m$ & $\lambda_{\text{theo.}}$ [nm] & $\lambda_{\text{beob.}}$ [nm] & Abweichung \\
      \hline
      4  & $1170.366 \pm 1.302$ & -     & -     \\
      5  & $621.527 \pm 0.692$  & $622.34 \pm 1$ & $0.67\sigma$ \\
      6  & $518.112 \pm 0.577$  & $517.50 \pm 1$ & $0.53\sigma$ \\
      7  & $477.124 \pm 0.531$  & $474.00 \pm 1$ & $2.76\sigma$ \\
      8  & $456.100 \pm 0.508$  & $454.40 \pm 1$ & $1.52\sigma$ \\
      9  & $443.719 \pm 0.494$  & $445.30 \pm 1$ & $1.42\sigma$ \\
      \hline
  \end{tabular}
  \label{tab:wellenlaengen_2ns_zsmf}
\end{table}

Noch einmal nach dem gleichen Schema berechneten wir abschließend die erwarteten Wellenlängen der Hauptserie. Hierzu bestimmten wir zunächst den Korrekturterm
\begin{align}
  \Delta_p = 0.8803 \pm 0.0007.
\end{align}

Die berechneten Wellenlängen für die Quantenzahlen $m=4,5$ sind in \tabref{tab:wellenlaengen_hs_zsfm} zu finden. Da beide unter $350\si{\nano\meter}$ liegen und damit außerhalb des von uns beobachteten Bereichs, können wir diese nicht vergleichen.

\begin{table}[H]
  \centering
  \caption{Berechnete Wellenlängen der Hauptserie $mp \to 3s$.}
  \vspace*{0.5em}
  \begin{tabular}{c c c c}
      \hline
      $m$ & $\lambda_{\text{theo.}}$ [nm] & $\lambda_{\text{beob.}}$ [nm] & Abweichung \\
      \hline
      4  & $332.4 \pm 0.6$ & -     & -     \\
      5  & $286.6 \pm 0.5$  & - & - \\
      \hline
  \end{tabular}
  \label{tab:wellenlaengen_hs_zsfm}
\end{table}

Im letzten Teil der Auswertung passten wir die Funktion
\begin{align}
  f(m;E_{Ry}, E_{3p}, \Delta_{d(s)}) = \frac{hc}{\frac{E_{Ry}}{(m - \Delta_{d(s)})^2} - E_{3p}} = \lambda_m
\end{align}
abhängig von der Quantenzahl $m$ an die Werte der beobachteten Wellenlängen der 1. und 2. Nebenserie an, um die Parameter $E_{Ry}, E_{3p}$ und $\Delta_{d}$ bzw. $\Delta_{s}$ zu bestimmen. Die optimierten Parameter aus der Anpassung an die 1. Nebenserie sind in \tabref{tab:fit_1ns_vergl} zusammengefasst. Zusätzlich ist hier auch die Abweichung zu den Werten, welche wir im vorherigen Aufgabenteil berechnet hatten, angegeben.


\begin{table}[H]
  \centering
  \caption{Werte der Parameter, bestimmt durch Fit an die Wellenlängen der 1. Nebenserie, und Vergleich.}
  \vspace*{0.5em}
  \begin{tabular}{c c c c}
      \hline
      Parameter & Wert & Fehler & Abw. vom zuvor berechneten Wert \\
      \hline
      $E_{Ry}$  & $-13.0 \si{\electronvolt}$ & $0.5\si{\electronvolt}$ & $1.53\sigma$ \\
      $E_{3p}$  & $-3.023\si{\electronvolt}$  & $0.006\si{\electronvolt}$ & $0.75\sigma$ \\
      $\Delta_{d}$  & $0.07$  & $0.05$ & - \\
      \hline
  \end{tabular}
  \label{tab:fit_1ns_vergl}
\end{table}

Die reduzierte $\chisq$-Summe liegt mit einem Wert von $\chisqrd = 1.50$ nahe am optimalen Wert $1$. Somit können wir bei diesen Fit die Beschreibung der Daten durch das gegebene Modell bzw. die gegebene Funktion allgemein als gut Bewerten. Dies ist auch durch relativ geringen Unsicherheiten der optimierten Parameter zu erkennen. Die Fitwahrscheinlichkeit liegt allerdings nur bei $16.0\%$. Wie aber auch in der Praktikumsanleitung angeführt, liegt dies daran, dass $\Delta_d$ nur eine empirische Näherungsformel ist und in Wirklichkeit auch schwach von der Hauptquantenzahl $n$ abhängt. Die Abweichungen von den zuvor berechneten Werten sind mit Werten von um einem $\sigma$ nicht ausschlaggebend.

Im Anschluss führten wir noch einen Fit an die Wellenlängen der 2. Nebenserie durch. Hierbei optimierten wir anstatt dem Korrekturterm $\Delta_d$, den Term $\Delta_s$, welchen wir auch bereits im vorherigen Aufgabenteil verwendet hatten. Die optimierten Werte der Parameter sind, in gleicher Form wie zuvor, in \tabref{tab:fit_2ns_vergl} zusammengefasst.

\begin{table}[H]
  \centering
  \caption{Werte der Parameter, bestimmt durch Fit an die Wellenlängen der 2. Nebenserie, und Vergleich.}
  \vspace*{0.5em}
  \begin{tabular}{c c c c}
      \hline
      Parameter & Wert & Fehler & Abw. vom zuvor berechneten Wert \\
      \hline
      $E_{Ry}$  & $-11.6 \si{\electronvolt}$ & $2.1\si{\electronvolt}$ & $1.01\sigma$ \\
      $E_{3p}$  & $-3.01\si{\electronvolt}$  & $0.04\si{\electronvolt}$ & $0.68\sigma$ \\
      $\Delta_{s}$  & $1.62$  & $0.25$ & $1.03\sigma$ \\
      \hline
  \end{tabular}
  \label{tab:fit_2ns_vergl}
\end{table}

Für diesen Fit hatten wir nur halb so viele Datenpunkte wie im vorherigen Fit zur Verfügung. Dies spiegelt sich auch in den nun sehr hohen Ungenauigkeiten der optimierten Parameter, sowie der Bewertung der Güte des Fits wider. Die reduzierte $\chisq$-Summe beträgt hier $\chisqrd = 4.28$, was für eine schlechte Beschreibung der Daten durch das verwendete Modell steht. Mit gerade einmal $1\%$ ist auch hier die Fitwahrscheinlichkeit sehr schlecht, was erneut auf das zum vorherigen Fit beschriebene Problem zurückzuführen sein könnte. Die Abweichungen zu den zuvor berechneten Werten der optimierten Parameter sind in diesem Fit geringer als im vorherigen, was allerdings daran liegt, dass die Unsicherheiten hier deutlich größer sind.