\section{Auswertung}

Bevor wir mit der Aufzeichnung der eigentlichen Spektren begonnen haben, haben wir eine Dunkelstrommessung durchgeführt. Das Spektrum dieser ist in \abbref{fig:dunkelstrommessung} zu sehen.

\begin{figure}[H]
  \centering
  \includegraphics[width=.9\textwidth]{files/pngs/dunkelstrommessung.png}
  \caption{Spektrum Dunkelstrommessung}
  \label{fig:dunkelstrommessung}  
\end{figure}

Es handelt sich hierbei um ein Untergrundrauschen, welches wir mit der \texttt{OceanView} Software automatisch von allen weiteren aufgezeichneten Spektren abziehen.

\subsection{Untersuchung des Sonnenlichtspektrums}

Am Tag der Versuchsdurchführung war das Wetter leider stark bewölkt, weshalb wir mit dem Spektroskop nicht auf den blauen Himmel zielen konnten. Es ist also zu beachten, dass die folgenden Betrachtungen durch die Auswirkungen der Wolkendecke gestört sind. Die untenstehende \abbref{fig:himmel_m_o_g} zeigt aufgezeichnete Spektrum des Tageslichts durch das geöffnete Fenster (blau) und durch die Fensterscheibe (orange) im Vergleich. Es ist bereits hier zu sehen, dass die Intensität durch das Glas über das gesamte Spektrum hinweg abgeschwächt wird. Die stärkste Abschwächung verzeichnen wir bei niedrigen Wellenlängen, also im UV-Bereich. Dies ist auch am Verlauf der Absorption in \abbref{fig:absorption_glas} zu sehen. Die geringste Abschwächung tritt im Bereich der Wellenlänge zwischen $400$ bis $600\si{\nano\meter}$, also dem sichtbaren Bereich auf. In Richtung des Rot- bis Infrarotbereichs steigt die Absorption wieder etwas an.

\begin{figure}[H]
  \centering
  \includegraphics[width=0.8\textwidth]{files/plots/himmel_m_o_g.png}
  \caption{Sonnen}
  \label{fig:himmel_m_o_g}
\end{figure}

\begin{figure}[H]
  \centering
  \includegraphics[width=0.8\textwidth]{files/plots/absorption_glas.png}
  \caption{Absorption}
  \label{fig:absorption_glas}
\end{figure}

Die vielen im Spektrum sichtbaren lokalen Minima sind gerade die Wellenlängen der Frauenhoferlinien, welche durch Absorption von Licht bestimmter Wellenlängen in der Sonnen- und Erdatmosphäre entstehen. \abbref{fig:spektrum_frauenhofer_balmer} zeigt erneut das Spektrum des Sonnenlichts, ohne Fensterscheibe. Markiert sind hier in Orange nun die Minima im Spektrum, welche jeweils am nächsten an der erwarteten Wellenlänge einer Frauenhoferlinie liegen. Zusätzlich sind in Grün die Literaturwerte der Wellenlänge der Balmer-Serie von Wasserstoff eingezeichnet.

\begin{figure}[H]
  \centering
  \includegraphics[width=\textwidth]{files/plots/spektrum_frauenhofer_balmer.png}
  \caption{Frauenhoferlinien und Balmerserie}
  \label{fig:spektrum_frauenhofer_balmer}
\end{figure}

\tabref{tab:frauenhofer_vergleich} zeigt eine Aufschlüsselung der erwarteten Wellenlänge der Frauenhoferlinien, die von uns aus dem Spektrum abgelesenen werte, sowie die Abweichung zwischen den Werten. Als Fehler für die abgelesenen Wellenlängen haben wir hier einen Wert von $\pm 1 \si{\nano\meter}$ verwendet. Es ist zu sehen, dass sich die Abweichung, bis auf wenige ausnahmen auf unter einem $\sigma$ beläuft.

\begin{table}[h]
  \centering
  \caption{Vergleich der erwarteten und gemessenen Wellenlängen der Frauenhoferlinien}
  \vspace*{0.5em}
  \begin{tabular}{c|c|c|c}
      \hline
      Linie & Literaturwert [nm] & Abgelesener Wert [nm] & Abweichung [$\sigma$] \\
      \hline
      K  & 393.4 & 393.0 & 0.4 \\
      H  & 396.8 & 396.1 & 0.7 \\
      G  & 430.8 & 429.8 & 1.0 \\
      F  & 486.1 & 485.2 & 0.91 \\
      b1 & 518.4 & 516.7 & 1.7 \\
      E  & 527.0 & 526.2 & 0.8 \\
      D3 & 587.6 & 588.4 & 0.8 \\
      D2 & 589.0 & 589.0 & 0.0 \\
      D1 & 589.6 & 589.7 & 0.11 \\
      C  & 656.3 & 655.0 & 1.3 \\
      B  & 686.7 & 686.7 & 0.0 \\
      A  & 759.4 & 759.4 & 0.0 \\
      \hline
  \end{tabular}
  \label{tab:frauenhofer_vergleich}
\end{table}