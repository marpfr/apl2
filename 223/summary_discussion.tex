\section{Zusammenfassung und Diskussion}

Die Brown'schen Bewegung beschreibt die zufälligen Zickzackbewebungen kleiner Teilchen in einer Flüssigkeit oder einem Gas, die durch Stöße mit den umgebenden Molekülen verursacht werden. Sie wurde 1827 von Robert Brown beobachtet und später durch die kinetische Gastheorie und Albert Einsteins mathematische Beschreibung erklärt. Bei Betrachtung des mittleren Verschiebungsquadrats, welches ein Partikel durch die Brown'sche Bewegung zurücklegt ergibt sich ein direkter mathematischer Zusammenhang zwischen diesem und der Boltzmannkonstante, nach der Formel

\begin{gather*}
  k = \frac{6\pi\eta a}{4 T t} \expval{r^2}.
\end{gather*}

Ziel von Versuch 223 war es, die Boltzmannkonstante anhand der Brown'schen Bewegung von in Flüssigkeit suspendierten Latexpartikeln zu bestimmen. 