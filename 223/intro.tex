Bereits im Jahr 1827 war dem schottischen Botaniker Robert Brown aufgefallen, dass Blütenpollen in einem Glas Wasser eigenartige Zickzackbewebungen ausführen. Erst nahezu 100 Jahre später erkannte Albert Einstein, dass die Bewegungen auf fortwährende Stöße der Wassermoleküle zurückzuführen sind. Während Einstein damit ein weiteres Argument für die Existenz von Atomen und Molekülen lieferte, war seine Beobachtung auch eine Evidenz für die molekulare Theorie der Wärme. Die mittlere Geschwindigkeit der Wassermoleküle hängt demnach von der Temperatur des Wassers ab. Erst im Jahre 1926 konnte der französische Physiker Jean-Baptiste Perrin die Brown'sche Molekularbewegung experimentell mit hoher Genauigkeit bestätigen, wofür er auch den Physik-Nobelpreis erhielt.

In Versuch 223 beobachten wir mit einem Mikroskop die Brown'sche Bewegung von in Wasser suspendierten Partikeln. Wir werden die statistische Bewegung untersuchen und anhand der Bahn eines einzelnen Teilchens die Boltzmannkonstante bestimmen.

\subsection{Physikalische Grundlagen}

Als vereinfachtes Modell der Brown'schen Bewegung betrachten wir einen eindimensionalen Random-Walk. Wir stellen uns also ein einzelnes, freies Teilchen vor, welches sich auf einer Linie bewegen kann. Alls $\tau$ Sekunden erfährt das Teilchen einen Stoß, also $n = \frac{t}{\tau}$ Stöße in einer Zeit $t$. Bei jedem Stoß besteht die gleiche Wahrscheinlichkeit $p = \frac{1}{2}$, dass das Teilchen um die Distanz $\delta$ entweder nach links ($-\delta$) oder nach rechts ($+\delta$) verschoben wird. Um nach $n$ Stößen an der Position $m\delta$ zu sein, muss sich das Teilchen somit $\frac{n+m}{2}$-mal nach rechts und $\frac{n-m}{2}$-mal nach links bewegt haben. Um die Position $x = m\delta$ zu erreichen gibt es
\begin{align}
  \binom{n}{\frac{1}{2}(n+m)} = \frac{n!}{\qty(\frac{1}{2}(n+m))!\qty(\frac{1}{2}(n-m))!}
\end{align}
Wege, die das Teilchen gelaufen sein könnte. Die Wahrscheinlichkeit, dass sich das Teilchen nach $n$ Stößen an der Position $x = m\delta$ befindet ist binomialverteilt nach
\begin{align}
  P(m;n) = \binom{n}{\frac{1}{2}(n+m)} p^{\frac{n+m}{2}} (1-p)^{\frac{n-m}{2}}.
\end{align}
Die Wahrscheinlichkeit $p$ für einen Sprung nach links bzw. nach rechts ist für beide Richtungen $p = \frac{1}{2}$. Somit können wir die Wahrscheinlichkeitsfunktion vereinfachen zu
\begin{align}
  P(m;n) = \frac{n!}{\qty(\frac{1}{2}(n+m))!\qty(\frac{1}{2}(n-m))}\qty(\frac{1}{2})^n.
\end{align}

Da das Zeitintervall $\tau$ zwischen den Stößen sehr klein ist, wird $n = \frac{t}{\tau}$ sehr groß. Somit können $n!$ und $m!$ mit der Stirling'schen Formel
\begin{align*}
  n! &= (2\pi n)^{\frac{1}{2}} n^n \e{-n}
  \intertext{angenähert und die Wahrscheinlichkeitsfunktion zu}
  P(m;n) &= \sqrt{\frac{2}{\pi n}} \e{-\frac{m^2}{2n}}
\end{align*}
umgeformt werden.



\subsection{Versuchsdurchführung}



