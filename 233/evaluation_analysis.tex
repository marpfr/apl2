\section{Auswertung}

\subsection*{Vorbemerkungen}

Sofern nicht anders angegeben, berechnen wir die Fehler zusammengesetzter Werte anhand der standardmäßigen Gauß'schen Fehlerfortpflanzung. Die $\sigma$-Abweichung zweier fehlerbehafteter Werte $x \pm \Delta x$ und $y \pm \Delta y$ berechnen wir anhand der Formel
\begin{align}
  \sigma = \frac{\qty|x - y|}{\sqrt{\Delta x^2 + \Delta y^2}}.
\end{align}

Für die Breite des Analysierspalts, abgelesen von der Messuhr, verwedenen wir einen konstanten Fehler von $\pm0.01\si{\milli\meter}$. Für abgelesene Pixelwerte nehmen wir mit einem Fehler von $\pm 4$px an.

\subsection{Eichung der Abszisse}

Wir beginnen die Auswertung der Ergebnisse mit der Eichung der Abszisse, bestimmen also einen Faktor, um im weiteren Verlauf der Rechnungen Pixel in Millimeter umrechnen zu können. Hierzu tragen wir zunächst die Abstände der links- und rechtsseitigen Minima von der fünften bis zur ersten gegen die Breite des Analysierspalts, zu welcher diese gerade noch sichtbar waren, auf.

\begin{table}[H]
  \centering
  \caption{Abstände der Minima 1. bis 5. Ordnung mit zugehöriger Spaltbreite des Analysierspalts}
  \vspace*{0.5em}
  \begin{tabular}{c|c|c|c}
    Ordnung & Pixel (l) $\to$ Pixel (r) [px] & Abstand [px] & Spaltbreite [mm]\\\hline
    5 & $338 \pm 4 \to 1263 \pm 4$ & $925 \pm 6$ & $0.79 \pm 0.01$\\
    4 & $438 \pm 4 \to 1164 \pm 4$ & $726 \pm 6$  & $0.59 \pm 0.01$\\
    3 & $507 \pm 4 \to 1075 \pm 4$ & $568 \pm 6$ & $0.48 \pm 0.01$\\
    2 & $609 \pm 4 \to 984 \pm 4$ & $375 \pm 6$ & $0.30 \pm 0.01$\\
    1 & $689 \pm 4 \to 890 \pm 4$ & $201 \pm 6$ & $0.16 \pm 0.01$
  \end{tabular}
\end{table}

\abbref{fig:abszisseneichung} zeigt die Breiten des Analysierspalts über den jeweiligen Pixelwerten. An die Messdaten fitten wir eine standardmäßige lineare Funktion der Form
\begin{align}
  f(x;a,b) = ax + b.
\end{align}

Die aus dem Fit resultierenden optimierten Werte von $a$ und $b$ lauten
\begin{align}
  a &= (8.63 \pm 0.18) \cdot 10^{-4} \frac{\si{\milli\meter}}{\mathrm{px}},\\[1em]
  b &= -0.018 \pm 0.011 \si{\milli\meter}.
\end{align}

Die Steigung $a$ werden wir fortan als Umrechnungsfaktor verwenden, es gilt also
\begin{align}
  1\mathrm{px} = (8.63 \pm 0.18) \cdot 10^{-4}\si{\milli\meter}.
\end{align}
Zur Verbesserung der Genauigkeit verwenden wir hierbei nicht den hier angegebenen gerundeten Wert, sondern den ungerundeten, nur durch die Genauigkeit des \texttt{float}-Datentyps begrenzten Wert.

\begin{figure}[H]
  \centering
  \includegraphics[width=.9\textwidth]{files/plots/2/abszisseneichung.png}
  \caption{Pixel-Abstände über Spaltbreite zur Abszisseneichung}
  \label{fig:abszisseneichung}
\end{figure}
\newpage
\subsection{Quantitative Untersuchung der Beugung am Einzelspalt}

Wir entnehmen zunächst die Positionen der links- und rechtsseitigen Minima erster bis fünfter Ordnung aus den Intensitätsverteilungen. Zunächst aus der, in welcher das Hauptmaximum nicht in Sättigung ist (\abbref{fig:es_nicht_saett_extrema}), dann aus der Verteilung, in welcher das Hauptmaximum in Sättigung ist und die weiter außen liegenden Maxima ebenfalls sichtbar sind (\abbref{fig:es_saett_extrema}). Die Positionen sind in \tabref{tab:es_abst_minima} zusammengefasst. Außerdem ist hier direkt deren Abstand für die weiteren Berechnungen ausgerechnet.

\begin{table}[H]
  \centering
  \caption{Abstände der linksseitigen (l) und rechtsseitigen (r) Minima 1. bis 5. Ordnung}
  \vspace*{0.5em}
  \begin{tabular}{c|c|c}
    Ordnung & Pixel (l) $\to$ Pixel (r) [px] & Abstand [px]\\\hline
    5 & $324 \pm 4 \to 1264 \pm 4$ & $940 \pm 6$\\
    4 & $415 \pm 4 \to 1169 \pm 4$ & $754 \pm 6$\\
    3 & $510 \pm 4 \to 1075 \pm 4$ & $565 \pm 6$\\
    2 & $603 \pm 4 \to 979 \pm 4$ & $376 \pm 6$\\
    1 & $696 \pm 4 \to 885 \pm 4$ & $189 \pm 6$
  \end{tabular}
  \label{tab:es_abst_minima}
\end{table}

Zusätzlich entnehmen wir den Daten noch die Positionen der links- und rechtsseitigen Maxima erster bis fünfter Ordnung, gleichermaßen zusammengefasst in \tabref{tab:es_abst_maxima}.

\begin{table}[H]
  \centering
  \caption{Abstände der linksseitigen (l) und rechtsseitigen (r) Maxima 1. bis 5. Ordnung}
  \vspace*{0.5em}
  \begin{tabular}{c|c|c}
    Ordnung & Pixel (l) $\to$ Pixel (r) [px] & Abstand [px]\\\hline
    5 & $273 \pm 4 \to 1305 \pm 4$ & $1032 \pm 6$\\
    4 & $371 \pm 4 \to 1210 \pm 4$ & $839 \pm 6$\\
    3 & $464 \pm 4 \to 1118 \pm 4$ & $654 \pm 6$\\
    2 & $559 \pm 4 \to 1021 \pm 4$ & $462 \pm 6$\\
    1 & $660 \pm 4 \to 924 \pm 4$ & $264 \pm 6$
  \end{tabular}
  \label{tab:es_abst_maxima}
\end{table}

\begin{figure}[H]
  \centering
  \includegraphics[width=.9\textwidth]{files/plots/2/es_nicht_saett_extrema.png}
  \caption{Intensitätsverteilung bei Beugung am Einzelspalt mit Hauptmaximum nicht in Sättigung}
  \label{fig:es_nicht_saett_extrema}
\end{figure}

\begin{figure}[H]
  \centering
  \includegraphics[width=.9\textwidth]{files/plots/2/es_saett_extrema.png}
  \caption{Intensitätsverteilung bei Beugung am Einzelspalt mit Hauptmaximum in Sättigung}
  \label{fig:es_saett_extrema}
\end{figure}

Die Abstände der Minima tragen wir nun über der jeweiligen Ordnung in ein Diagramm auf, zu sehen in \abbref{fig:es_fit_ordnung_ohne_maxima}, und fitten an diese Datenpunkte erneut eine lineare Funktion (diesmal ohne y-Abschnitt, da es sich um eine Ursprungsgerade handelt), um die Steigung zu ermitteln. Hierbei erhalten wir den Wert
\begin{align}
  a &= 188.2 \pm 0.8 \si{\milli\meter}[1em].
\end{align}

Diesen können wir nun verwenden, um eine zwei weitere Aufgabenstellungen zu bearbeiten.

\begin{figure}[H]
  \centering
  \includegraphics[width=.9\textwidth]{files/plots/2/es_fit_ordnung_ohne_maxima.png}
  \caption{Ordnungen gegenüber der Abstände der jeweiligen Minima mit linearem Fit.}
  \label{fig:es_fit_ordnung_ohne_maxima}
\end{figure}

\subsubsection*{Berechnung der Spaltbreite}

Aus den Grundlagen der Beugung am Einzelspalt wissen wir, dass für den Winkel $\alpha_n$ eines Minimums $n$-ter Ordnung der Zusammenhang
\begin{align}
  b \cdot \sin(\alpha_n) = n \cdot \lambda
\end{align}
mit der Spaltbreite $b$ und der Wellenlänge $\lambda$ des einfallenden Lichts gilt. Weiter können wir aus geometrischen Überlegungen des Versuchsaufbaus herleiten, dass für die Position $x_n$ des $n$-ten Minimums auf dem Schirm in Abstand $d$
\begin{align}
  \tan(\alpha_n) = \frac{x_n}{d}
\end{align}
gilt. Da wir in unserem Aufbau den Schirm genau in der Brennweite $f$ der Sammellinse positioniert haben gilt für uns $d = f$. Für kleine $\alpha_n$ gilt $\sin(\alpha_n) \approx \alpha_n \approx \tan(\alpha_n)$, somit können wir die oberen beiden Gleichungen zusammenfassen zu
\begin{align}
  \frac{n\lambda}{b} = \frac{x_n}{f},
\end{align}
welche wir zur Spaltbreite $b$ umformen können:
\begin{align}
  b = \frac{f \lambda}{\frac{x_n}{n}}.
\end{align}
Der Bruch $\frac{x_n}{n}$ entspricht dabei genau der Steigung $a$ der Gerade, welche wir gerade eben an die Abstände der Minima gefittet haben. Somit haben wir mit
\begin{align}
  b = \frac{f\lambda}{a}
\end{align}
eine Formel für die Spaltbreite hergeleitet. In diese setzen wir die Wellenlänge des Laserlichts von $\lambda = 532 \cdot 10^{-6} \si{\milli\meter}$, die Brennweite $f = 80 \pm 2 \si{\milli\meter}$, sowie die zuvor bestimmte Steigung, welche wir zuvor mit dem Umrechnungsfaktor in Millimeter umrechnen, ein. Wir erhalten damit eine Spaltbreite von
\begin{align}
  b = (0.262 \pm 0.007)\si{\milli\meter}.
\end{align}

\subsubsection*{Bestimmung der Ordnungen der Nebenmaxima}

Das Verhältnis der Ordnungen der Nebenmaxima zu ihren Abständen sollte dem gleichen proportionalen Verhältnis folgen, wie das der Nebenminima. Um dies zu bestätigen, stellen wir das proportionale Verhältnis um, um vom Abstand der links- und rechtsseitigen Maxima auf ihre Ordnung schließen zu können. 

\begin{align}
  \mathrm{ord}_{\max} = \frac{\mathrm{Abstand}_{\max}}{\mathrm{Steigung}}
\end{align}
Die Resultate der Berechnungen sind in \tabref{tab:es_maxima_ord_ber_vergl} zusammengefasst und auch in \abbref{fig:es_fit_ordnung} gegen die Abstände aufgetragen. An den Zahlenwerten sehen wir, dass die Ordnungen immer in etwa zwischen den ganzen Zahlen der Ordnungen der Minima liegen, was sich auch grafisch in \abbref{fig:es_fit_ordnung} bestätigen lässt.

Um diese Werte noch mit den theoretischen Vorhersagen zu vergleichen, betrachten wir die Maxima der sinc-Funktion. Wir in den theoretischen Grundlagen erklärt, gilt für die Intensitätsverteilung des Beugungsbildes des Einzelspalts
\begin{gather}
  I(k_y) = F(k_y)^2 = \sinc(\frac{k_y d}{2})^2 d^2
  \intertext{mit Nullstellen bei}
  k_y = \frac{2\pi n}{d}.
\end{gather}
Setzen wir dies in die Gleichung oben ein, so erhalten wir
\begin{align}
  I(n) = \sinc(n\pi)^2 d^2,
\end{align}
wobei wir den Faktor $d^2$ vernachlässigen können, da es uns nur um die $x$-Positionen der Extrema geht. Die Maxima bestimmen wir, indem wir die Funktion in den Online-Grafikrechner \texttt{Desmos} eingeben und diese ablesen, wie in \abbref{fig:maxima_normed_sinc} zu sehen. Die Abweichung von den Berechneten werten bestimmen wir anhand der $\sigma$-Abweichung mit dem Fehler der berechneten Ordnung.

\begin{table}[H]
  \centering
  \caption{Abstände der linksseitigen (l) und rechtsseitigen (r) Maxima, die berechneten Ordnungen und Vergleich zu den theoretischen Vorhersagen.}
  \vspace*{0.5em}
  \begin{tabular}{c|c|c|c|c}
    Pixel (l) $\to$ Pixel (r) [px] & Abstand [px] & Ber. Ord. & Theo. Ord. & Abweichung\\\hline
    $273 \pm 4 \to 1305 \pm 4$ & $1032 \pm 6$ & $5.48 \pm 0.04$ & $5.48$ & $0$\\
    $371 \pm 4 \to 1210 \pm 4$ & $839 \pm 6$ & $4.46 \pm 0.04$ & $4.48$ & $0.5\sigma$\\
    $464 \pm 4 \to 1118 \pm 4$ & $654 \pm 6$ & $3.48 \pm 0.04$ & $3.47$ & $0.25\sigma$\\
    $559 \pm 4 \to 1021 \pm 4$ & $462 \pm 6$ & $2.46 \pm 0.04$ & $2.46$ & $0$\\
    $660 \pm 4 \to 924 \pm 4$ & $264 \pm 6$ & $1.40 \pm 0.04$ & $1.43$ & $0.75\sigma$
  \end{tabular}
  \label{tab:es_maxima_ord_ber_vergl}
\end{table}

\begin{figure}[H]
  \centering
  \includegraphics[width=.9\textwidth]{files/plots/2/es_fit_ordnung.png}
  \caption{Ordnungen gegenüber der Abstände der jeweiligen Minima mit linearem Fit und berechnete Ordnungen der Maxima.}
  \label{fig:es_fit_ordnung}
\end{figure}

\begin{figure}[H]
  \centering
  \includegraphics[width=\textwidth]{files/plots/2/maxima_normed_sinc.png}
  \caption{Maxima der normierten sinc-Funktion.}
  \label{fig:maxima_normed_sinc}
\end{figure}

\subsubsection*{Vergleich der Intensitäten}

An dieser Stelle der Auswertung würden wir die Intensitäten der Maxima in den Aufgezeichneten Beugungsbildern mit den theoretisch erwarteten Beugungsbildern vergleichen. Wie allerdings bereits auf den Abbildungen \ref{fig:es_nicht_saett_extrema} und \ref{fig:es_saett_extrema} zu sehen ist, ist uns bei der Aufzeichnung der Intensitätsverteilungen ein Fehler unterlaufen, sodass die Maxima 0. und 1. Ordnung bereits sehr stark in Sättigung sind. Möglicherweise war hier die Belichtungszeit bereits bei der Aufnahme mit der Kamera zu hoch eingestellt, oder wir hatten beim Export über \texttt{Gwyddeon} eine falsche Einstellung gewählt. 

Theoretisch würden wir wie folgt vorgehen: Anhand der Intensität 0. Maximums und der Beachtung der verschiedenen Belichtungszeiten können wir die beiden Intensitätsverteilung normieren, sodass wir einen verhältnismäßigen Abstieg der Maxima 1. bis 5. Ordnung im Vergleich zum Maximum 0. Ordnung erhalten.

Dann generieren wir ein theoretisches Beugungsbild anhand des in der Praktikumsanleitung bereitgestellten Skripts, wie sie in \abbref{fig:es_theorie_beugungsbild} zu sehen ist.

\begin{figure}[H]
  \centering
  \includegraphics[width=.9\textwidth]{files/plots/2/es_theorie_beugungsbild.png}
  \caption{Theoretisches Beugungsbild des Einzelspalts.}
  \label{fig:es_theorie_beugungsbild}
\end{figure}

In diesem Beugungsmuster sind die Intensitäten ebenfalls anhand der Intensität des Maximums 0. Ordnung normiert. Nun können wir diese auslesen, entweder mit numerischen Methoden in Python oder wieder anhand von einem Online-Grafikrechner, und anschließend mit den gemessenen, normierten Intensitäten vergleichen.

\subsection{Quantitative Untersuchung der Beugung am Doppelspalt}

Einleitend zu dieser Aufgabe haben wir qualitativ die Auswirkungen verschiedener Geometrien des Doppelspalts betrachtet. Dazu waren auf dem Dia drei verschiedene Doppelspalte, in drei verschiedenen Breiten, angebracht.

Auf dem Beugungsbild des breitesten Doppelspalts, zu sehen in \abbref{fig:breit} konnten wir deutlich das mittlere Hauptmaximum mit drei bis view Nebenmaxima, sowie viele weitere Hauptmaxima mit jeweils etwa ein bis zwei Nebenmaxima beobachten.

\begin{figure}[H]
  \centering
  \includegraphics[width=.9\textwidth]{files/3/breit.png}
  \caption{Beugungsbild des breiten Doppelspalts.}
  \label{fig:breit}
\end{figure}


Das Beugungsbild des schmalen Doppelspalts (\abbref{fig:schmal}) zeigte ein sehr breites Hauptmaximum 0. Ordnung, ebenfalls mit etwa drei bis vier Nebenmaxima. Allerdings waren hier fast keine weiteren Hauptmaxima höherer Ordnung zu sehen.

\begin{figure}[H]
  \centering
  \includegraphics[width=.9\textwidth]{files/3/schmal.png}
  \caption{Beugungsbild des schmalen Doppelspalts.}
  \label{fig:schmal}
\end{figure}


Das Beugungsbild des mittelgroßen Doppelspalts (\abbref{fig:mittel}), bestand aus einem etwas schmaleren Hauptmaximum 0. Ordnung mit etwa drei Nebenmaxima und weiteren weniger gut sichtbaren Hauptmaxima mit etwa einem Nebenmaximum. 

\begin{figure}[H]
  \centering
  \includegraphics[width=.9\textwidth]{files/3/mittel.png}
  \caption{Beugungsbild des mittleren Doppelspalts.}
  \label{fig:mittel}
\end{figure}

Mit dem mittelgroßen Doppelspalt haben wir auch die weiteren Messungen für diese Aufgabe durchgeführt.

\abbref{fig:ds_theorie_beugungsbild} zeigt das theoretische Beugungsbild des Doppelspalts, generiert mit dem in der Praktikumsanleitung gegebenen Python-Skript. Wir verwenden hierfür den Spaltabstand und die Spaltbreite, wie wir sie in Aufgabe 5 bestimmt haben.

\begin{figure}[H]
  \centering
  \includegraphics[width=.9\textwidth]{files/plots/3/ds_theorie_beugungsbild.png}
  \caption{Theoretisches Beugungsbild des Doppelspalts mit Funktion des Einzelspalts und Gitterfunktion.}
  \label{fig:ds_theorie_beugungsbild}
\end{figure}

Es ist hier deutlich zu sehen, wie die einhüllende Funktion des Einzelspalts maßgeblich die Form des Beugungsbildes des Doppelspalts beeinflusst. Die Übergänge zwischen den Hauptmaxima bilden sich immer dort, wo sowohl die Gitterfunktion, als auch die Einzelspaltfunktion eine Nullstelle besitzen. Minima innerhalb der Hauptmaxima bilden sich durch Nullstellen der Gitterfunktion, während die Einzelspaltfunktion größer Null ist.

Im Vergleich mit dem gemessenen Beugungsbild zeigen sich, wie zu erwarten, sehr ähnliche Strukturen. Dieses ist in \abbref{fig:ds_gemessen_beugungsbild} zu sehen.

\begin{figure}[H]
  \centering
  \includegraphics[width=.9\textwidth]{files/plots/3/ds_gemessen_beugungsbild.png}
  \caption{Gemessenes Beugungsbild des Doppelspalts.}
  \label{fig:ds_gemessen_beugungsbild}
\end{figure}

Auch hier sehen wir die fünf Maxima, welche gemeinsam zum mittleren Hauptmaximum 0. Ordnung gehören. Darauf folgt ein breiteres Minimum, welches den Übergang zum Hauptmaximum 1. Ordnung darstellt. In diesem finden sich, wie im theoretischen Bild, zwei kleinere Nebenmaxima.

Wir möchten nun, so wie bereits in Theorie für den Einzelspalt erklärt, die Intensitäten der Nebenmaxima innerhalb der Einhüllenden des nullten Hauptmaximums mit den theoretischen Werten vergleichen. Die Vergleichswerte ermitteln wir wieder numerisch mit dem Online-Grafikrechner, wie in \abbref{fig:ds_intensitaeten_desmos} zu sehen.

Um die Intensitäten vergleichen zu können, berechnen wir zunächst den Normierungsfaktor aus der Intensität des 0. Hauptmaximums
\begin{align}
  N = \frac{1}{I_{0,0}} = (9.78500 \pm 0.00957) \cdot 10^{-4}.
\end{align}

Mit diesem Wert multiplizieren wir nun die Intensität aller in \abbref{fig:ds_gemessen_beugungsbild} Maxima, um so die auf 1 normierten Intensitäten zu erhalten. Diese können wir dann mit den theoretischen Werten vergleichen. Die Resultate des Vergleichs sind in \tabref{tab:es_vergleich_intensitaet} aufgelistet.


\begin{figure}[H]
  \centering
  \includegraphics[width=.9\textwidth]{files/plots/3/ds_intensitaeten_desmos.png}
  \caption{Numerische Bestimmung der normierten Intensitäten des 0. Hauptmaximums des Doppelspalts.}
  \label{fig:ds_intensitaeten_desmos}
\end{figure}


\begin{table}[H]
  \centering
  \caption{Vergleich der Intensitäten der Nebenmaxima des 0. Hauptmaximums beim Einzelspalt.}
  \vspace*{0.5em}
  \begin{tabular}{|c|c|c|c|c|}\hline
    Ordnung & Intensität & Intensität (normiert) & Intensität (theo.) & Abw.\\\hline
    \multicolumn{5}{|l|}{Linksseitige Maxima}\\\hline
    1 & $628 \pm 2$   & $0.6146 \pm 0.0021$    &  $0.5417 \pm 0.0001$     &   $35.52\sigma$\\
    2 & $58 \pm 2$    & $0.0568 \pm 0.0020$    &  $0.0448 \pm 0.0001$     &   $6.1\sigma$\\\hline
    \multicolumn{5}{|l|}{Rechtsseitige Maxima}\\\hline
    1 & $616 \pm 2$  &  $0.6028 \pm 0.0021$    &  $0.5417 \pm 0.0001$   &     $29.84\sigma$\\
    2 & $55 \pm 2$   &  $0.0538 \pm 0.0020$    &  $0.0448 \pm 0.0001$   &     $4.61\sigma$\\\hline
  \end{tabular}
  \label{tab:es_vergleich_intensitaet}
\end{table}



%Normfaktor: (9.78500 \pm 0.00957)e-4
%O I          I (normiert)           I (theo)               Abw
%Linksseitige Maxima
%1 628 \pm 2    0.61450 \pm 0.00205      0.5417 \pm 0.0001        35.52σ
%2 58 \pm 2     0.05675 \pm 0.00196      0.0448 \pm 0.0001        6.1σ
%Rechtsseitige Maxima
%1 616 \pm 2    0.60276 \pm 0.00204      0.5417 \pm 0.0001        29.84σ
%2 55 \pm 2     0.05382 \pm 0.00196      0.0448 \pm 0.0001        4.61σ

\newpage
\subsection{Das Objektbild als Fouriersynthese des Beugungsbildes am Beispiel des Einfachspalts}