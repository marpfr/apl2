\documentclass[a4paper,12pt]{article}
\usepackage[ngerman]{babel}
\usepackage{graphicx}
\usepackage{color}
\usepackage[utf8]{inputenc}
\usepackage{amsthm}
\usepackage{amsmath}
\usepackage{amssymb}
\usepackage{physics}
\usepackage{enumerate}
\usepackage{siunitx}
\usepackage{hyperref}
\usepackage{pdfpages}
\usepackage{pgfplots}
\usepackage{parskip}
\usepackage{tikz}
\usepackage{bm}
\usepackage{stackengine}
\usepackage{dsfont}
\usepackage{titlesec}
\usepackage{float}
\usepackage{multicol}
\usepackage{enumitem}
\usepackage{multirow}

% EINIGE COMMANDS

\newcommand{\R}[0]{\mathbb{R}}
\newcommand{\N}[0]{\mathbb{N}}
\newcommand{\C}[0]{\mathbb{C}}
\newcommand{\Z}[0]{\mathbb{Z}}
\newcommand{\Q}[0]{\mathbb{Q}}
\definecolor{lblue}{RGB}{51,153,255} % Farbe für Notizen und besondere Zeichen
\newcommand{\bbew}[0]{\bigg| \bigg|} % Um Endergebnisse doppel zu unterstreichen, außer dass die Striche nicht unten sondern zur Seite stehen
\newcommand{\baro}[0]{\textcolor{lblue}{| }} % Strich um Rechnung von Nebenrechnung/Kommentar zu trennen
\newcommand{\arrowroa}[0]{\textcolor{lblue}{\to}}
\newcommand{\ind}[0]{\ \ \ \ \ }
\newcommand{\indd}[0]{\ind \ind }
\newcommand{\evalint}[3]{\bqty{#1}_{#2}^{#3}} % Stammfunktion mit eckigen Klammern die automatisch auf die entsprechende Größe gesetzt werden
\newcommand{\diff}{\mathrm{\ d}} % d aber nicht in kursiv
\newcommand{\EQ}[0]{\mathrel{\stackon[1pt]{$=$}{$\scriptstyle!$}}} % Gleichheitszeichen mit ! oben drauf
\newcommand{\limit}[2]{\lim\limits_{#1 \to #2}}
\DeclareMathOperator{\Lagr}{\mathcal{L}} % Lagrnaginag
\newcommand{\ELG}[1]{\dv{}{t} \pdv{\Lagr}{\dot{#1}} - \pdv{\Lagr}{#1}} % ELG aber man muss nicht so viel schreiben
\def\tbs{\textbackslash} % backslash but shorter
\newcommand{\e}[1]{\text{e}^{#1}}
\newcommand{\opnorm}[1]{\norm{#1}_{\text{op}}}
\newcommand{\summ}[2]{\sum\limits_{#1}^{#2}}

\newcommand{\mean}[1]{\langle #1 \rangle}

\newcommand{\todo}[1]{\textcolor{red}{TODO: #1}}

\newcommand{\abbref}[1]{Abbildung (\ref{#1})}
\newcommand{\tabref}[1]{Tabelle (\ref{#1})}

\newcommand{\isot}[2]{$^{#1}\mathrm{#2}$}

% EINHEITEN

\newcommand{\gm}[0]{\text{ gm}}
\newcommand{\cm}[0]{\text{\,cm}}
\newcommand{\cmm}[0]{\text{ cm}}
\newcommand{\dm}[0]{\text{ dm}}
\newcommand{\km}[0]{\text{ km}}
\newcommand{\kg}[0]{\text{ kg}}
\newcommand{\m}[0]{\text{ m}}
\newcommand{\s}[0]{\text{ s}}
\newcommand{\J}[0]{\text{ J}}
\newcommand{\Newt}[0]{\text{ N}}
\newcommand{\rad}[0]{\text{ rad}}
\newcommand{\mm}[0]{\text{ mm}}
\newcommand{\mbar}[0]{\text{ mbar}}
\newcommand{\barp}[0]{\text{ bar}}
\newcommand{\mPa}[0]{\text{ mPa}}
\newcommand{\Pa}[0]{\text{ Pa}}
\newcommand{\K}[0]{\text{ K}}
\renewcommand{\l}[0]{\text{ l}}
\newcommand{\mol}[0]{\text{ mol}}
\newcommand{\Mol}[0]{\text{Mol}}
\newcommand{\V}[0]{\text{ V}}
\newcommand{\A}[0]{\text{ A}}
\newcommand{\mA}[0]{\text{ mA}}
\newcommand{\muA}[0]{\ \mu\text{A}}
\newcommand{\Ohm}[0]{\ \Omega}
\newcommand{\kOhm}[0]{\ \text{k}\Omega}
\newcommand{\mOhm}[0]{\ \text{m}\Omega}
\newcommand{\Celsius}[0]{\text{ $^\circ$C}}
\newcommand{\const}[0]{\text{ const.}}

\newcommand{\idf}[0]{\mathrm{id}}

\newcommand{\err}[1]{\Delta #1}
\newcommand{\errpdvsq}[2]{\qty(\pdv{#1}{#2} \cdot \err{#2})^2}
\newcommand{\errsq}[2]{\qty(#1 \cdot \err{#2})^2}
\newcommand{\errsqval}[2]{\qty(#1 \cdot #2)^2}
\newcommand{\errpdv}[2]{\pdv{#1}{#2} \cdot \err{#2}}

\newcommand{\chisq}{\chi^{2}}
\newcommand{\chisqrd}{\chi^{2}_{\mathrm{red}}}

%LA Zeug
\newcommand{\EndVS}[0]{\mathrm{End}} % Endomorphismen
\newcommand{\EigVS}[0]{\mathrm{Eig}} % Eigenraum
\newcommand{\IsomVS}[0]{\mathrm{Isom}} %Isometrien
\newcommand{\Ann}[0]{\mathrm{Ann}} %Annullator
\newcommand{\LinSpan}[0]{\mathrm{Span}} %Linearer Spann
\newcommand{\GL}[0]{\mathrm{GL}} % General Linear Group
\newcommand{\Mat}[3]{\mathrm{Mat}(#1,#2;#3)} % Matritzenraum
\newcommand{\idmat}[0]{\mathds{1}} % einheitsmatrix symbol
\newcommand{\idmatd}[2]{\mathds{1}_{#1 \times #2}} % einheitsmatrix symbol mit suffix
\newcommand{\im}[0]{\mathrm{im}} % Bildraum
\newcommand{\id}[0]{\mathrm{id}} % Identitätsabbildung
\newcommand{\modsp}[2]{{\raisebox{.2em}{$#1$}\left/\raisebox{-.3em}{$#2$}\right.}} % Faktor/Quotient
\newcommand{\longto}[0]{\longrightarrow} %langes äquivalent zu \to
\newcommand{\Hau}[2]{\mathrm{Hau_{#1}\qty(#2)}}

\newcommand{\zz}[0]{Z\kern-.3em\raise-0.5ex\hbox{Z}: } % zu zeigen / z.z.



% MATHEMATISCHE OPERATOREN DIE NICHT IN DEN PACKAGES SIND

\DeclareMathOperator{\arcosh}{arcosh}
\DeclareMathOperator{\acsinh}{arsinh}
\DeclareMathOperator{\artanh}{artanh}
\DeclareMathOperator{\arcsech}{arcsech}
\DeclareMathOperator{\arccsch}{arcCsch}
\DeclareMathOperator{\arccoth}{arcCoth}


\setlength{\textheight}{26cm} \setlength{\footskip}{1cm}
\setlength{\topmargin}{-20mm} \setlength{\headheight}{5mm}
\setlength{\headsep}{5mm} \setlength{\textwidth}{16cm}
\setlength{\evensidemargin}{0mm} \setlength{\oddsidemargin}{0mm}


% NÜTZLICHE TRICKS

% Benutze kein \left( \right), der "physics" package hat \pqty{} (runde Klammern), \bqty{} (eckige Klammern), die die Größe der Klammern automatisch anpassen mehr zum phyisics package hier: http://mirrors.ibiblio.org/CTAN/macros/latex/contrib/physics/physics.pdf

% 


\def\checkmark{\tikz\fill[scale=0.4](0,.35) -- (.25,0) -- (1,.7) -- (.25,.15) -- cycle;} 

\DeclareSIPrePower\quartic{4}
\sisetup{per-mode=fraction}
\sisetup{exponent-product=\cdot}
\renewcommand{\labelenumi}{\alph{enumi})}
\setlength{\parindent}{0pt}
\newcounter{versuchnr}
\newcounter{gruppe}
\allowdisplaybreaks

\titleformat*{\section}{\large\bfseries}
\titleformat*{\subsection}{\normalsize\bfseries}
\titleformat*{\subsubsection}{\normalsize\it}

\renewcommand{\arraystretch}{1.2}





\begin{document}
	
	%%%%%%% Name(n) der Abgebenden
	\newcommand{\name}{Marius Pfeiffer}

	%%%%%%% Daten Versuch
	\setcounter{versuchnr}{241}
	\newcommand{\versuchtitel}{Wechselstromeigenschften von RLC-Gliedern}
	\newcommand{\protocolpdf}{files/241messprotokoll.pdf}
	\newcommand{\titleimg}{files/versuchsaufbau.png}
	\newcommand{\betreuer}{Kristian Köhler}
	\newcommand{\datum}{17.12.2024 \& 13.01.2025}
	
	
	\begin{center}
		{Physikalisches Anfängerpraktikum II für Lehramtsstudierende}\\[3mm]
	\end{center}
	
	\vspace*{5mm}
	
	\noindent Name: \name{}
	\hfill Matrikel-Nr.: 4188573\\
	E-Mail: marius.pfeiffer@stud.uni-heidelberg.de
	
	\vspace*{2mm}
	\noindent Betreut durch: \betreuer
    \hfill \datum

	\vspace*{2mm}
	\hrule \vspace*{10mm}

  \begin{center}
    {\bf \large Versuch \arabic{versuchnr}: \it \bf \versuchtitel}
  \end{center}

  \vspace*{10mm}

  \begin{figure}[H]
    \centering
    \includegraphics[width=.8\textwidth]{\titleimg}
    \caption{Versuchsaufbau}
  \end{figure}

  \tableofcontents
  \newpage\noindent
	
	%%%%%%%%%%%%%%%%%%%%%%%%%%%%%%%%%%%%%%%%%%%%%%%%%%%%%%%%%%%%%%%%%%%%%%%%
	\section{Einleitung}
	
	In Versuch 255 setzen wir uns mit der Funktionsweise einer Röntgenröhre, sowie dem charakteristischen Spektrum der Röntenstrahlung auseinander. Neben quantitativen Untersuchungen des Röntgenspektrums selbst, nutzen wir das Prinzip der Bragg-Reflexion, um unter anderem die Gitterkonstante eines NaCl-Kristalls zu bestimmen.

\subsection{Physikalische Grundlagen}

\subsubsection*{Die Röntgenröhre}
Eine Röntgenröhre ist aufgebaut aus einer Glühkathode und einer Anode, welche sich in einem evakuierten Glaskolben befinden. Durch Glühemmission werden aus der Kathode Elektronen freigesetzt, welche durch eine Beschleunigungsspannung von $10$ bis $100\si{\kilo\volt}$, welche zwischen Kathode und Anode anliegt beschleunigt werden. Der kontinuierliche Teil des Röntgenspektrums wird durch die ausgehende Bremsstrahlung beim Abbremsen der Elektronen im Anodenmaterial verursacht. Diese Strahlung setzt bei einer bestimmten Grenzwellenlänge $\lambda_{gr}$ ein, welche sich nach
\begin{align}
  \lambda_{gr} = \frac{hc}{eU}
\end{align}
berechnen lässt. Hierbei sind $h,c,e$ das Plank'sche Wirkungsquantum, die Lichtgeschwindigkeit und die Elementarladung. $U$ ist die an der Röntgenröhre anliegende Beschleunigungsspannung. Durch frei werdende Strahlung bei der Ionisation des Anodenmaterials ist dem kontinuierlichen Spektrum ein diskretes Spektrum überlagert, zu sehen in \abbref{fig:roentgenspektrum}, welches charakteristisch für das jeweilige Anodenmaterial ist. 

\begin{figure}[H]
  \centering
  \includegraphics[width=.75\textwidth]{files/roentgenspektrum.png}
  \caption{Kontinuierliches und diskretes Röntgenspektrum.}
  \label{fig:roentgenspektrum}
\end{figure}

Die Positionen der Linien im diskreten Spektrum sind abhängig von der Ursprungs- und Ziel-Schale von bzw. zu welcher der Übergang des Elektrons stattfindet. Beispielsweise bezeichnen wir die Strahlung der Übergänge von der $L$- auf die $K$-Schale als $K_{\alpha}$-Strahlung, die für die Übergänge der $M$- auf die $K$-Schale als $K_{\beta}$-Strahlung. Die freiwerdende Energie eines Übergangs von der $n$-ten zur $m$-ten Schale lässt sich durch das Moseley'sche Gesetz
\begin{align}
  E_{n\to m} = hc R_{\infty} (Z-A)^2 \qty(\frac{1}{m^2} - \frac{1}{n^2})
\end{align}
berechnen. Hier gehen die Rydbergkonstante $R_{\infty}$, sowie die Kernladungszahl $Z$ und die Abschirmung der Kernladung als Abschirmungskonstante $A$ mit ein. Nähert man die Abschirmungskonstante als $A \approx 1$ an, so lässt sich mit dem Moseley'schen Gesetz eine näherung der Energie für die $K_{\alpha}$-Strahlung abhängig von der Kernladungszahl angeben:
\begin{align}
  E_{2 \to 1} = hc R_{\infty} (Z - 1)^2 \qty(\frac{1}{1}- \frac{1}{2^2}) = \frac{3}{4} hc R_{\infty} (Z - 1)^2.
\end{align}

Es ist zu beachten, dass bei genauerer Betrachtung, neben der Hauptquantenzahl, noch eine Entartung der Drehimpuls- und Spinquantenzahl in die freigesetzte Energie der Übergänge eingeht. Das Moseley'sche Gesetz gibt somit nur eine Näherung an.

\subsubsection*{Bragg-Reflexion}

Als Bragg-Reflexion bezeichnet man die Beugung von Röntenstrahlung durch die Gitterstruktur von Kristallen. Die Atomabstände in der Kristallstruktur befinden sich in der gleichen Größenordnung wie die Wellenlängen der Röntenstrahlung, weshalb sich die Bragg-Reflexion zur Untersuchung des Röntgenspektrums eignet. Die Bragg-Reflexion beruht darauf, dass auf den Kristall treffende Röntgenstrahlung sowohl an der Oberfläche, als auch den tieferliegenden Netzebenen der Kristallstruktur reflektiert wird. Beträgt der Gangunterschied $\Delta s = 2 d \sin(\vartheta)$ zweier Teilbündel, die unter dem Winkel $\vartheta$ eintreffen, siehe \abbref{fig:bragg_reflexion_drehkristall_a}, ein Vielfaches der Wellenlänge $\lambda$, so interferieren diese Konstruktiv, andernfalls destruktiv. Dies ist festgehalten durch das Bragg'sche Gesetz
\begin{align}
  2 d \sin(\vartheta) = n \lambda, \quad n \in \N.
\end{align}

\begin{figure}[H]
  \centering
  \includegraphics[width=.9\textwidth,trim={2cm 14cm 0.5cm 0},clip]{files/bragg_reflexion_drehkristall.png}
  \caption{Bragg-Reflexion von Röntenstrahlung an einem Kristall.}
  \label{fig:bragg_reflexion_drehkristall_a}
\end{figure}

Bei der sogenannten Drehkristallmethode wird diese Gesetzmäßigkeit angewandt. Dabei trifft Röntenstrahlung auf einen Kristall, welcher um die Achse senkrecht zu dieser gedreht wird. Welche Wellenlänge reflektiert wird, hängt dann gerade von der Winkelstellung des Kristalls ab. Die Intensität der reflektierten Strahlung kann dann mit einem Zählrohr gemessen und so Spektren wie in \abbref{fig:roentgenspektrum} aufgezeichnet werden. Ist der Drehkristall so justiert, dass gerade die Wellenlänge einer bekannten diskreten Linie reflektiert wird, lässt sich mit dem Bragg'schen Gesetz der Netzabstand $d$ berechnen.

\subsubsection*{Kristallstruktur}

Kristalle sind aus sich periodisch wiederholenden Elementarzellen aufgebaut, wie sie Beispielsweise in \abbref{fig:elementarzelle_nacl_reordered} zu sehen sind. Bei NaCl- und LiF-Kristallen sind die drei Gitterkonstanten, also die Seitenlängen $a$ einer Elementarzelle, gleich groß.

\begin{figure}[H]
  \centering
  \includegraphics[width=\textwidth]{files/elementarzelle_nacl_reordered.png}
  \caption{Elementarzelle eines NaCl-Kristalls, deren periodische Anordnung, sowie der Kristallschnitt.}
  \label{fig:elementarzelle_nacl_reordered}
\end{figure}

Zur Ermittlung der Anzahl an Atomen, die einer Elementarzelle angehören, muss man beachten, zu wie vielen benachbarten Elementarzellen die Atome jeweils außerdem zählen. Dies ist ebenfalls in \abbref{fig:elementarzelle_nacl_reordered} dargestellt. Trägt ein Atom zu $x$ Elementarzellen bei, so zählt es für eine einzelne Elementarzelle nur zu $\flatfrac{1}{x}$. So lässt sich die Anzahl an Atomen am Beispiel der NaCl- (LiF-) Kristalle bestimmen.

\begin{table}[H]
  \centering
  \begin{tabular}{c|c|c|c||c|c}
    Element & Position & Beitrag zu x EZ & Beitrag zu einer EZ & Anzahl & Beitrag\\\hline
    Cl (F) & Ecken & $8$ & $\flatfrac{1}{8}$ & 8 & 1\\
    Cl (F) & Mitte Fläche & $2$ & $\flatfrac{1}{2}$ & 6 & 3\\
    Na (Li) & Mitte Kante & $4$ & $\flatfrac{1}{4}$ & 12 & 3\\
    Na (Li) & Mitte Zelle & $1$ & $\flatfrac{1}{1}$ & 1 & 1\\
  \end{tabular}
\end{table}

Aus den Beiträgen können ablesen, dass zu jeder Elementarzelle $4$ Chlor- (Fluor-) Atome und $4$ Natrium- (Lithium-) Atome, also $4$ NaCl- (LiF-) Moleküle, beitragen. Die Zahl $4$ geht bei der Berechnung der Avogadrokonstante wie folgt mit ein:
\begin{align}
  N_{A} = 4 \frac{V_{Mol}}{V}.
\end{align}
Dabei ist $V_{Mol}$ das Molvolumen eines einzelnen Moleküls und $V$ das Volumen einer Elementarzelle. Dieses lässt sich aus dem Netzabstand $d$, welcher hier $\flatfrac{a}{2}$ beträgt, berechnen. Somit gilt
\begin{align}
  N_A = 4 \frac{V_{Mol}}{(2d)^3} = 4 \frac{M_{Mol}}{\rho (2d)^3} = \frac{1}{2}\frac{M_{Mol}}{\rho d^3}.
\end{align}

\subsection{Versuchsdurchführung}

Zur Durchführung der Messungen verwenden wir ein Röntgengerät, welches mit einem Zählrohr-Goniometer ausgestattet ist (\abbref{fig:goniometer}). Die Röntgenstrahlung wird durch eine Röntgenröhre mit Molybdänanode erzeugt und über einen Kollimator auf den Drehkristall fokussiert. Das Zählrohr am Goniometer dreht sich im Verhältnis 2:1 mit dem Drehkristall, sodass es immer genau die Strahlung, welche im Winkel $\vartheta$ reflektiert wird erfassen kann.

\begin{figure}[H]
  \centering
  \includegraphics[width=.8\textwidth]{files/goniometer.png}
  \caption{Montierung des Drehkristalls und des Zählrohr-Goniometers im Winkelverhältnis 2:1.}
  \label{fig:goniometer}
\end{figure}

\textbf{Messung des Röntgenspektrums mit einem LiF-Kristall.} Bei einer Beschleunigungsspannung von $U = 35 \si{\kilo\volt}$, einem Strom von $I = 1\si{\milli\ampere}$ und einer Torzeit von $t = 5\si{\second}$ zeichnen wir das Röntgenspektrum in einem Winkelbereich von $3\si{\degree}$ bis $22\si{\degree}$ in $0.2\si{\degree}$-Schritten auf.

\textbf{Vermessen der $K_{\alpha}$ und $K_{\beta}$-Linien des Anodenmaterials.} Aus unseren Messdaten der vorherigen Aufgabe entnehmen wir die ungefähren Positionen der $K_{\alpha}$- und $K_{\beta}$-Linien. Mit den gleichen Einstellungen für Beschleunigungsspannung und Strom nehmen wir in deren Umgebung das Spektrum noch einmal in Winkelschritten von $0.1\si{\degree}$ und einer Torzeit von $20 \si{\second}$ auf.

\textbf{Zählrate als Funktion der Beschleunigungsspannung.} Bei einem festen Winkel von $7.5\si{\degree}$ Zeichnen wir die Zählrate für Beschleunigungsspannungen im Bereich von $20$ bis $35\si{\kilo\volt}$ in $1\si{\kilo\volt}$-Schritten jeweils über $20\si{\second}$ auf.

\textbf{Messung des Röntgenspektrums mit einem NaCl-Kristall.} Hier führen wir die Messung aus Aufgabe 1 noch einmal mit einem NaCl-Kristall in einem Winkelbereich von $3\si{\degree}$ bis $18\si{\degree}$ durch.
	
	
  
	%%%%%%%%%%%%%%%%%%%%%%%%%%%%%%%%%%%%%%%%%%%%%%%%%%%%%%%%%%%%%%%%%%%%%%%%
	\includepdf[scale=0.83,pages=1,pagecommand=\section{Messprotokoll}]{\protocolpdf}
  \includepdf[scale=0.83,pages=2-,pagecommand={\thispagestyle{plain}}]{\protocolpdf}
	
	%%%%%%%%%%%%%%%%%%%%%%%%%%%%%%%%%%%%%%%%%%%%%%%%%%%%%%%%%%%%%%%%%%%%%%%%
	\newpage\noindent
	\section{Auswertung}

\subsection*{Vorbemerkungen}

Sofern nicht anders angegeben, berechnen wir die Fehler zusammengesetzter Werte anhand der standardmäßigen Gauß'schen Fehlerfortpflanzung. Die $\sigma$-Abweichung zweier fehlerbehafteter Werte $x \pm \Delta x$ und $y \pm \Delta y$ berechnen wir anhand der Formel
\begin{align}
  \sigma = \frac{\qty|x - y|}{\sqrt{\Delta x^2 + \Delta y^2}}.
\end{align}

Für die Breite des Analysierspalts, abgelesen von der Messuhr, verwedenen wir einen konstanten Fehler von $\pm0.01\si{\milli\meter}$. Für abgelesene Pixelwerte nehmen wir mit einem Fehler von $\pm 4$px an.

\subsection{Eichung der Abszisse}

Wir beginnen die Auswertung der Ergebnisse mit der Eichung der Abszisse, bestimmen also einen Faktor, um im weiteren Verlauf der Rechnungen Pixel in Millimeter umrechnen zu können. Hierzu tragen wir zunächst die Abstände der links- und rechtsseitigen Minima von der fünften bis zur ersten gegen die Breite des Analysierspalts, zu welcher diese gerade noch sichtbar waren, auf.

\begin{table}[H]
  \centering
  \caption{Abstände der Minima 1. bis 5. Ordnung mit zugehöriger Spaltbreite des Analysierspalts}
  \vspace*{0.5em}
  \begin{tabular}{c|c|c|c}
    Ordnung & Pixel (l) $\to$ Pixel (r) [px] & Abstand [px] & Spaltbreite [mm]\\\hline
    5 & $338 \pm 4 \to 1263 \pm 4$ & $925 \pm 6$ & $0.79 \pm 0.01$\\
    4 & $438 \pm 4 \to 1164 \pm 4$ & $726 \pm 6$  & $0.59 \pm 0.01$\\
    3 & $507 \pm 4 \to 1075 \pm 4$ & $568 \pm 6$ & $0.48 \pm 0.01$\\
    2 & $609 \pm 4 \to 984 \pm 4$ & $375 \pm 6$ & $0.30 \pm 0.01$\\
    1 & $689 \pm 4 \to 890 \pm 4$ & $201 \pm 6$ & $0.16 \pm 0.01$
  \end{tabular}
\end{table}

\abbref{fig:abszisseneichung} zeigt die Breiten des Analysierspalts über den jeweiligen Pixelwerten. An die Messdaten fitten wir eine standardmäßige lineare Funktion der Form
\begin{align}
  f(x;a,b) = ax + b.
\end{align}

Die aus dem Fit resultierenden optimierten Werte von $a$ und $b$ lauten
\begin{align}
  a &= (8.63 \pm 0.18) \cdot 10^{-4} \frac{\si{\milli\meter}}{\mathrm{px}},\\[1em]
  b &= -0.018 \pm 0.011 \si{\milli\meter}.
\end{align}

Die Steigung $a$ werden wir fortan als Umrechnungsfaktor verwenden, es gilt also
\begin{align}
  1\mathrm{px} = (8.63 \pm 0.18) \cdot 10^{-4}\si{\milli\meter}.
\end{align}
Zur Verbesserung der Genauigkeit verwenden wir hierbei nicht den hier angegebenen gerundeten Wert, sondern den ungerundeten, nur durch die Genauigkeit des \texttt{float}-Datentyps begrenzten Wert.

\begin{figure}[H]
  \centering
  \includegraphics[width=.9\textwidth]{files/plots/2/abszisseneichung.png}
  \caption{Pixel-Abstände über Spaltbreite zur Abszisseneichung}
  \label{fig:abszisseneichung}
\end{figure}
\newpage
\subsection{Quantitative Untersuchung der Beugung am Einzelspalt}

Wir entnehmen zunächst die Positionen der links- und rechtsseitigen Minima erster bis fünfter Ordnung aus den Intensitätsverteilungen. Zunächst aus der, in welcher das Hauptmaximum nicht in Sättigung ist (\abbref{fig:es_nicht_saett_extrema}), dann aus der Verteilung, in welcher das Hauptmaximum in Sättigung ist und die weiter außen liegenden Maxima ebenfalls sichtbar sind (\abbref{fig:es_saett_extrema}). Die Positionen sind in \tabref{tab:es_abst_minima} zusammengefasst. Außerdem ist hier direkt deren Abstand für die weiteren Berechnungen ausgerechnet.

\begin{table}[H]
  \centering
  \caption{Abstände der linksseitigen (l) und rechtsseitigen (r) Minima 1. bis 5. Ordnung}
  \vspace*{0.5em}
  \begin{tabular}{c|c|c}
    Ordnung & Pixel (l) $\to$ Pixel (r) [px] & Abstand [px]\\\hline
    5 & $324 \pm 4 \to 1264 \pm 4$ & $940 \pm 6$\\
    4 & $415 \pm 4 \to 1169 \pm 4$ & $754 \pm 6$\\
    3 & $510 \pm 4 \to 1075 \pm 4$ & $565 \pm 6$\\
    2 & $603 \pm 4 \to 979 \pm 4$ & $376 \pm 6$\\
    1 & $696 \pm 4 \to 885 \pm 4$ & $189 \pm 6$
  \end{tabular}
  \label{tab:es_abst_minima}
\end{table}

Zusätzlich entnehmen wir den Daten noch die Positionen der links- und rechtsseitigen Maxima erster bis fünfter Ordnung, gleichermaßen zusammengefasst in \tabref{tab:es_abst_maxima}.

\begin{table}[H]
  \centering
  \caption{Abstände der linksseitigen (l) und rechtsseitigen (r) Maxima 1. bis 5. Ordnung}
  \vspace*{0.5em}
  \begin{tabular}{c|c|c}
    Ordnung & Pixel (l) $\to$ Pixel (r) [px] & Abstand [px]\\\hline
    5 & $273 \pm 4 \to 1305 \pm 4$ & $1032 \pm 6$\\
    4 & $371 \pm 4 \to 1210 \pm 4$ & $839 \pm 6$\\
    3 & $464 \pm 4 \to 1118 \pm 4$ & $654 \pm 6$\\
    2 & $559 \pm 4 \to 1021 \pm 4$ & $462 \pm 6$\\
    1 & $660 \pm 4 \to 924 \pm 4$ & $264 \pm 6$
  \end{tabular}
  \label{tab:es_abst_maxima}
\end{table}

\begin{figure}[H]
  \centering
  \includegraphics[width=.9\textwidth]{files/plots/2/es_nicht_saett_extrema.png}
  \caption{Intensitätsverteilung bei Beugung am Einzelspalt mit Hauptmaximum nicht in Sättigung}
  \label{fig:es_nicht_saett_extrema}
\end{figure}

\begin{figure}[H]
  \centering
  \includegraphics[width=.9\textwidth]{files/plots/2/es_saett_extrema.png}
  \caption{Intensitätsverteilung bei Beugung am Einzelspalt mit Hauptmaximum in Sättigung}
  \label{fig:es_saett_extrema}
\end{figure}

Die Abstände der Minima tragen wir nun über der jeweiligen Ordnung in ein Diagramm auf, zu sehen in \abbref{fig:es_fit_ordnung_ohne_maxima}, und fitten an diese Datenpunkte erneut eine lineare Funktion (diesmal ohne y-Abschnitt, da es sich um eine Ursprungsgerade handelt), um die Steigung zu ermitteln. Hierbei erhalten wir den Wert
\begin{align}
  a &= 188.2 \pm 0.8 \si{\milli\meter}[1em].
\end{align}

Diesen können wir nun verwenden, um eine zwei weitere Aufgabenstellungen zu bearbeiten.

\begin{figure}[H]
  \centering
  \includegraphics[width=.9\textwidth]{files/plots/2/es_fit_ordnung_ohne_maxima.png}
  \caption{Ordnungen gegenüber der Abstände der jeweiligen Minima mit linearem Fit.}
  \label{fig:es_fit_ordnung_ohne_maxima}
\end{figure}

\subsubsection*{Berechnung der Spaltbreite}

Aus den Grundlagen der Beugung am Einzelspalt wissen wir, dass für den Winkel $\alpha_n$ eines Minimums $n$-ter Ordnung der Zusammenhang
\begin{align}
  b \cdot \sin(\alpha_n) = n \cdot \lambda
\end{align}
mit der Spaltbreite $b$ und der Wellenlänge $\lambda$ des einfallenden Lichts gilt. Weiter können wir aus geometrischen Überlegungen des Versuchsaufbaus herleiten, dass für die Position $x_n$ des $n$-ten Minimums auf dem Schirm in Abstand $d$
\begin{align}
  \tan(\alpha_n) = \frac{x_n}{d}
\end{align}
gilt. Da wir in unserem Aufbau den Schirm genau in der Brennweite $f$ der Sammellinse positioniert haben gilt für uns $d = f$. Für kleine $\alpha_n$ gilt $\sin(\alpha_n) \approx \alpha_n \approx \tan(\alpha_n)$, somit können wir die oberen beiden Gleichungen zusammenfassen zu
\begin{align}
  \frac{n\lambda}{b} = \frac{x_n}{f},
\end{align}
welche wir zur Spaltbreite $b$ umformen können:
\begin{align}
  b = \frac{f \lambda}{\frac{x_n}{n}}.
\end{align}
Der Bruch $\frac{x_n}{n}$ entspricht dabei genau der Steigung $a$ der Gerade, welche wir gerade eben an die Abstände der Minima gefittet haben. Somit haben wir mit
\begin{align}
  b = \frac{f\lambda}{a}
\end{align}
eine Formel für die Spaltbreite hergeleitet. In diese setzen wir die Wellenlänge des Laserlichts von $\lambda = 532 \cdot 10^{-6} \si{\milli\meter}$, die Brennweite $f = 80 \pm 2 \si{\milli\meter}$, sowie die zuvor bestimmte Steigung, welche wir zuvor mit dem Umrechnungsfaktor in Millimeter umrechnen, ein. Wir erhalten damit eine Spaltbreite von
\begin{align}
  b = (0.262 \pm 0.007)\si{\milli\meter}.
\end{align}

\subsubsection*{Bestimmung der Ordnungen der Nebenmaxima}

Das Verhältnis der Ordnungen der Nebenmaxima zu ihren Abständen sollte dem gleichen proportionalen Verhältnis folgen, wie das der Nebenminima. Um dies zu bestätigen, stellen wir das proportionale Verhältnis um, um vom Abstand der links- und rechtsseitigen Maxima auf ihre Ordnung schließen zu können. 

\begin{align}
  \mathrm{ord}_{\max} = \frac{\mathrm{Abstand}_{\max}}{\mathrm{Steigung}}
\end{align}
Die Resultate der Berechnungen sind in \tabref{tab:es_maxima_ord_ber_vergl} zusammengefasst und auch in \abbref{fig:es_fit_ordnung} gegen die Abstände aufgetragen. An den Zahlenwerten sehen wir, dass die Ordnungen immer in etwa zwischen den ganzen Zahlen der Ordnungen der Minima liegen, was sich auch grafisch in \abbref{fig:es_fit_ordnung} bestätigen lässt.

Um diese Werte noch mit den theoretischen Vorhersagen zu vergleichen, betrachten wir die Maxima der sinc-Funktion. Wir in den theoretischen Grundlagen erklärt, gilt für die Intensitätsverteilung des Beugungsbildes des Einzelspalts
\begin{gather}
  I(k_y) = F(k_y)^2 = \sinc(\frac{k_y d}{2})^2 d^2
  \intertext{mit Nullstellen bei}
  k_y = \frac{2\pi n}{d}.
\end{gather}
Setzen wir dies in die Gleichung oben ein, so erhalten wir
\begin{align}
  I(n) = \sinc(n\pi)^2 d^2,
\end{align}
wobei wir den Faktor $d^2$ vernachlässigen können, da es uns nur um die $x$-Positionen der Extrema geht. Die Maxima bestimmen wir, indem wir die Funktion in den Online-Grafikrechner \texttt{Desmos} eingeben und diese ablesen, wie in \abbref{fig:maxima_normed_sinc} zu sehen. Die Abweichung von den Berechneten werten bestimmen wir anhand der $\sigma$-Abweichung mit dem Fehler der berechneten Ordnung.

\begin{table}[H]
  \centering
  \caption{Abstände der linksseitigen (l) und rechtsseitigen (r) Maxima, die berechneten Ordnungen und Vergleich zu den theoretischen Vorhersagen.}
  \vspace*{0.5em}
  \begin{tabular}{c|c|c|c|c}
    Pixel (l) $\to$ Pixel (r) [px] & Abstand [px] & Ber. Ord. & Theo. Ord. & Abweichung\\\hline
    $273 \pm 4 \to 1305 \pm 4$ & $1032 \pm 6$ & $5.48 \pm 0.04$ & $5.48$ & $0$\\
    $371 \pm 4 \to 1210 \pm 4$ & $839 \pm 6$ & $4.46 \pm 0.04$ & $4.48$ & $0.5\sigma$\\
    $464 \pm 4 \to 1118 \pm 4$ & $654 \pm 6$ & $3.48 \pm 0.04$ & $3.47$ & $0.25\sigma$\\
    $559 \pm 4 \to 1021 \pm 4$ & $462 \pm 6$ & $2.46 \pm 0.04$ & $2.46$ & $0$\\
    $660 \pm 4 \to 924 \pm 4$ & $264 \pm 6$ & $1.40 \pm 0.04$ & $1.43$ & $0.75\sigma$
  \end{tabular}
  \label{tab:es_maxima_ord_ber_vergl}
\end{table}

\begin{figure}[H]
  \centering
  \includegraphics[width=.9\textwidth]{files/plots/2/es_fit_ordnung.png}
  \caption{Ordnungen gegenüber der Abstände der jeweiligen Minima mit linearem Fit und berechnete Ordnungen der Maxima.}
  \label{fig:es_fit_ordnung}
\end{figure}

\begin{figure}[H]
  \centering
  \includegraphics[width=\textwidth]{files/plots/2/maxima_normed_sinc.png}
  \caption{Maxima der normierten sinc-Funktion.}
  \label{fig:maxima_normed_sinc}
\end{figure}

\subsubsection*{Vergleich der Intensitäten}

An dieser Stelle der Auswertung würden wir die Intensitäten der Maxima in den Aufgezeichneten Beugungsbildern mit den theoretisch erwarteten Beugungsbildern vergleichen. Wie allerdings bereits auf den Abbildungen \ref{fig:es_nicht_saett_extrema} und \ref{fig:es_saett_extrema} zu sehen ist, ist uns bei der Aufzeichnung der Intensitätsverteilungen ein Fehler unterlaufen, sodass die Maxima 0. und 1. Ordnung bereits sehr stark in Sättigung sind. Möglicherweise war hier die Belichtungszeit bereits bei der Aufnahme mit der Kamera zu hoch eingestellt, oder wir hatten beim Export über \texttt{Gwyddeon} eine falsche Einstellung gewählt. 

Theoretisch würden wir wie folgt vorgehen: Anhand der Intensität 0. Maximums und der Beachtung der verschiedenen Belichtungszeiten können wir die beiden Intensitätsverteilung normieren, sodass wir einen verhältnismäßigen Abstieg der Maxima 1. bis 5. Ordnung im Vergleich zum Maximum 0. Ordnung erhalten.

Dann generieren wir ein theoretisches Beugungsbild anhand des in der Praktikumsanleitung bereitgestellten Skripts, wie sie in \abbref{fig:es_theorie_beugungsbild} zu sehen ist.

\begin{figure}[H]
  \centering
  \includegraphics[width=.9\textwidth]{files/plots/2/es_theorie_beugungsbild.png}
  \caption{Theoretisches Beugungsbild des Einzelspalts.}
  \label{fig:es_theorie_beugungsbild}
\end{figure}

In diesem Beugungsmuster sind die Intensitäten ebenfalls anhand der Intensität des Maximums 0. Ordnung normiert. Nun können wir diese auslesen, entweder mit numerischen Methoden in Python oder wieder anhand von einem Online-Grafikrechner, und anschließend mit den gemessenen, normierten Intensitäten vergleichen.

\subsection{Quantitative Untersuchung der Beugung am Doppelspalt}

Einleitend zu dieser Aufgabe haben wir qualitativ die Auswirkungen verschiedener Geometrien des Doppelspalts betrachtet. Dazu waren auf dem Dia drei verschiedene Doppelspalte, in drei verschiedenen Breiten, angebracht.

Auf dem Beugungsbild des breitesten Doppelspalts, zu sehen in \abbref{fig:breit} konnten wir deutlich das mittlere Hauptmaximum mit drei bis view Nebenmaxima, sowie viele weitere Hauptmaxima mit jeweils etwa ein bis zwei Nebenmaxima beobachten.

\begin{figure}[H]
  \centering
  \includegraphics[width=.9\textwidth]{files/3/breit.png}
  \caption{Beugungsbild des breiten Doppelspalts.}
  \label{fig:breit}
\end{figure}


Das Beugungsbild des schmalen Doppelspalts (\abbref{fig:schmal}) zeigte ein sehr breites Hauptmaximum 0. Ordnung, ebenfalls mit etwa drei bis vier Nebenmaxima. Allerdings waren hier fast keine weiteren Hauptmaxima höherer Ordnung zu sehen.

\begin{figure}[H]
  \centering
  \includegraphics[width=.9\textwidth]{files/3/schmal.png}
  \caption{Beugungsbild des schmalen Doppelspalts.}
  \label{fig:schmal}
\end{figure}


Das Beugungsbild des mittelgroßen Doppelspalts (\abbref{fig:mittel}), bestand aus einem etwas schmaleren Hauptmaximum 0. Ordnung mit etwa drei Nebenmaxima und weiteren weniger gut sichtbaren Hauptmaxima mit etwa einem Nebenmaximum. 

\begin{figure}[H]
  \centering
  \includegraphics[width=.9\textwidth]{files/3/mittel.png}
  \caption{Beugungsbild des mittleren Doppelspalts.}
  \label{fig:mittel}
\end{figure}

Mit dem mittelgroßen Doppelspalt haben wir auch die weiteren Messungen für diese Aufgabe durchgeführt.

\abbref{fig:ds_theorie_beugungsbild} zeigt das theoretische Beugungsbild des Doppelspalts, generiert mit dem in der Praktikumsanleitung gegebenen Python-Skript. Wir verwenden hierfür den Spaltabstand und die Spaltbreite, wie wir sie in Aufgabe 5 bestimmt haben.

\begin{figure}[H]
  \centering
  \includegraphics[width=.9\textwidth]{files/plots/3/ds_theorie_beugungsbild.png}
  \caption{Theoretisches Beugungsbild des Doppelspalts mit Funktion des Einzelspalts und Gitterfunktion.}
  \label{fig:ds_theorie_beugungsbild}
\end{figure}

Es ist hier deutlich zu sehen, wie die einhüllende Funktion des Einzelspalts maßgeblich die Form des Beugungsbildes des Doppelspalts beeinflusst. Die Übergänge zwischen den Hauptmaxima bilden sich immer dort, wo sowohl die Gitterfunktion, als auch die Einzelspaltfunktion eine Nullstelle besitzen. Minima innerhalb der Hauptmaxima bilden sich durch Nullstellen der Gitterfunktion, während die Einzelspaltfunktion größer Null ist.

Im Vergleich mit dem gemessenen Beugungsbild zeigen sich, wie zu erwarten, sehr ähnliche Strukturen. Dieses ist in \abbref{fig:ds_gemessen_beugungsbild} zu sehen.

\begin{figure}[H]
  \centering
  \includegraphics[width=.9\textwidth]{files/plots/3/ds_gemessen_beugungsbild.png}
  \caption{Gemessenes Beugungsbild des Doppelspalts.}
  \label{fig:ds_gemessen_beugungsbild}
\end{figure}

Auch hier sehen wir die fünf Maxima, welche gemeinsam zum mittleren Hauptmaximum 0. Ordnung gehören. Darauf folgt ein breiteres Minimum, welches den Übergang zum Hauptmaximum 1. Ordnung darstellt. In diesem finden sich, wie im theoretischen Bild, zwei kleinere Nebenmaxima.

Wir möchten nun, so wie bereits in Theorie für den Einzelspalt erklärt, die Intensitäten der Nebenmaxima innerhalb der Einhüllenden des nullten Hauptmaximums mit den theoretischen Werten vergleichen. Die Vergleichswerte ermitteln wir wieder numerisch mit dem Online-Grafikrechner, wie in \abbref{fig:ds_intensitaeten_desmos} zu sehen.

Um die Intensitäten vergleichen zu können, berechnen wir zunächst den Normierungsfaktor aus der Intensität des 0. Hauptmaximums
\begin{align}
  N = \frac{1}{I_{0,0}} = (9.78500 \pm 0.00957) \cdot 10^{-4}.
\end{align}

Mit diesem Wert multiplizieren wir nun die Intensität aller in \abbref{fig:ds_gemessen_beugungsbild} Maxima, um so die auf 1 normierten Intensitäten zu erhalten. Diese können wir dann mit den theoretischen Werten vergleichen. Die Resultate des Vergleichs sind in \tabref{tab:es_vergleich_intensitaet} aufgelistet.


\begin{figure}[H]
  \centering
  \includegraphics[width=.9\textwidth]{files/plots/3/ds_intensitaeten_desmos.png}
  \caption{Numerische Bestimmung der normierten Intensitäten des 0. Hauptmaximums des Doppelspalts.}
  \label{fig:ds_intensitaeten_desmos}
\end{figure}


\begin{table}[H]
  \centering
  \caption{Vergleich der Intensitäten der Nebenmaxima des 0. Hauptmaximums beim Einzelspalt.}
  \vspace*{0.5em}
  \begin{tabular}{|c|c|c|c|c|}\hline
    Ordnung & Intensität & Intensität (normiert) & Intensität (theo.) & Abw.\\\hline
    \multicolumn{5}{|l|}{Linksseitige Maxima}\\\hline
    1 & $628 \pm 2$   & $0.6146 \pm 0.0021$    &  $0.5417 \pm 0.0001$     &   $35.52\sigma$\\
    2 & $58 \pm 2$    & $0.0568 \pm 0.0020$    &  $0.0448 \pm 0.0001$     &   $6.1\sigma$\\\hline
    \multicolumn{5}{|l|}{Rechtsseitige Maxima}\\\hline
    1 & $616 \pm 2$  &  $0.6028 \pm 0.0021$    &  $0.5417 \pm 0.0001$   &     $29.84\sigma$\\
    2 & $55 \pm 2$   &  $0.0538 \pm 0.0020$    &  $0.0448 \pm 0.0001$   &     $4.61\sigma$\\\hline
  \end{tabular}
  \label{tab:es_vergleich_intensitaet}
\end{table}



%Normfaktor: (9.78500 \pm 0.00957)e-4
%O I          I (normiert)           I (theo)               Abw
%Linksseitige Maxima
%1 628 \pm 2    0.61450 \pm 0.00205      0.5417 \pm 0.0001        35.52σ
%2 58 \pm 2     0.05675 \pm 0.00196      0.0448 \pm 0.0001        6.1σ
%Rechtsseitige Maxima
%1 616 \pm 2    0.60276 \pm 0.00204      0.5417 \pm 0.0001        29.84σ
%2 55 \pm 2     0.05382 \pm 0.00196      0.0448 \pm 0.0001        4.61σ

\newpage
\subsection{Das Objektbild als Fouriersynthese des Beugungsbildes am Beispiel des Einfachspalts}
	
	
	
	%%%%%%%%%%%%%%%%%%%%%%%%%%%%%%%%%%%%%%%%%%%%%%%%%%%%%%%%%%%%%%%%%%%%%%%%
	\newpage\noindent
	\section{Zusammenfassung und Diskussion}

In Versuch 256 untersuchten wir das Phänomen der Röntgenfluoreszenz. Dies ist eine sekundäre Röntgenstrahlung welche abgestrahlt wird, wenn durch einfallende Röntgen-strahlung in einem Material Elektronenübergange hervorgerufen werden. Das abgestrahlte Spektrum zeigt die aus dem vorherigen Versuch bekannten $K_{\alpha}$- und $K_{\beta}$-Linien auf, welche charakteristisch für das bestrahlte Material sind. Angenähert folgt die bei einem Elektronenübergang auf den Schalen $n_2 \to n_1$ abgestrahlte Energie dem Moseleyschen Gesetz
\begin{align}
  \sqrt{E} = \sqrt{E_R} (Z - \sigma_{n_1,n_2}) \qty(\frac{1}{n_1^2} - \frac{1}{n_2^2}).
\end{align}

Im ersten Versuchsteil bestimmten wir die Rydberg-Energie $E_R$, sowie die Abschirmkonstante $\sigma_{12}$ für Übergänge $2 \to 1$, also $K_{\alpha}$-Übergänge, und  $\sigma_{13}$ für Übergänge $3 \to 1$, also $K_{\beta}$-Übergänge. Hierzu nahmen wir zunächst die Röntgenfluoreszenzspektren verschiedener reiner Metalle auf. Durch die Analysesoftware wurden die gemessenen abgestrahlten Energieimpulse entsprechend ihrer Stärke in Kanälen kategorisiert. Um damit rechnen zu können, mussten wir die Energieskala kalibrieren, indem wir die bekannten Energien der $K_{\alpha}$-Übergänge von Eisen und Molybdän mit den entsprechenden Kanälen, in welchen wir die Peaks fanden assoziieren. In \tabref{tab:elemente_kalph_kbeta_vergleich} sind die ermittelten Energiewerte der Linien zusammen mit den Literaturwerten\footnote{Quelle Literaturwerte Energien: https://physics.nist.gov/PhysRefData/XrayTrans/Html/search.html} zum Vergleich zu finden.


\begin{table}[H]
  \centering
  \begin{tabular}{|c|c|c|c|c|c|c|c|}
      \hline
      \# & Element & $E_{\alpha}\ [\si{\kilo\electronvolt}]$ & $E_{\alpha,\text{Lit}}\ [\si{\kilo\electronvolt}]$ & Abw. & $E_{\beta}\ [\si{\kilo\electronvolt}]$ & $E_{\beta,\text{Lit}}\ [\si{\kilo\electronvolt}]$ & Abw.  \\
      \hline
      1  & Mo  & $17.46 \pm 0.18$ & $-$ & $-$ & $19.57 \pm 0.17$ & $19.61$ & $0.24\sigma$\\
      2  & Fe     & $6.38  \pm 0.17$ & $-$ & $-$ & $7.03  \pm 0.42$ & $7.06$ & $0.08\sigma$\\
      3  & Ni    & $7.46  \pm 0.18$ & $7.48$ & $0.12\sigma$ & $8.26  \pm 0.21$ & $8.27$ & $0.05\sigma$\\
      4  & Zn      & $8.64  \pm 0.18$ & $8.64$ & $0$ & $9.59  \pm 0.16$ & $9.57$ & $0.13\sigma$\\
      5  & Zr & $15.78 \pm 0.18$ & $15.77$ & $0.06\sigma$ & $17.67 \pm 0.19$ & $17.67$ & $0$\\
      6  & Ti     & $4.44  \pm 0.19$ & $4.51$ & $0.37\sigma$ & $4.44  \pm 0.19$ & $4.93$ & $2.58\sigma$\\
      7  & Cu    & $8.04  \pm 0.17$ & $8.05$ & $0.06\sigma$ & $8.91  \pm 0.14$ & $8.91$ & $0$\\
      8  & Ag    & $21.89 \pm 0.21$ & $22.16$ & $1.29\sigma$ & $24.59 \pm 0.18$ & $24.94$ & $1.95\sigma$\\
      \hline
  \end{tabular}
  \caption{Energien der $K_{\alpha}$- und $K_{\beta}$-Linien der untersuchten Elemente und Vergleich mit den entsprechenden Literaturwerten.}
  \label{tab:elemente_kalph_kbeta_vergleich}
\end{table}

Nacheinander trugen wir einem die $K_{\alpha}$-Energien und einmal die $K_{\beta}$-Energien über der Kernladungszahl $Z$ der Elemente in einem Diagramm auf und passten die oben genannte Funktion an diese Daten an.
Aus der Anpassung an die $K_{\alpha}$-Energie ermittelten wir so die Werte
\begin{align}
  \sigma_{12} &= 1.4 \pm 0.5
\intertext{und}
  E_R &= (14.1 \pm 0.4) \si{\electronvolt}.
\end{align}

Im Vergleich mit dem Wert von etwa $13.6\si{\electronvolt}$ aus der Praktikumsanleitung weicht der von uns berechnete Wert um etwa $1.54\sigma$ ab. Der ermittelte Wert von $\sigma_{12}$ weicht um etwa $1.03\sigma$ vom Literaturwert\footnote{Quelle Literaturwerte $\sigma_{12}$, $\sigma_{13}$: https://de.wikipedia.org/wiki/Moseleysches\_Gesetz} von ungefähr $1.0$ ab.

Aus der Anpassung der Funktion an die $K_{\beta}$-Energien konnten wir die Werte
\begin{align}
  \sigma_{13} &= 2.2 \pm 0.4
\intertext{und}
  E_R &= (13.87 \pm 0.21) \si{\electronvolt}.
\end{align}
bestimmen.

Der hier für $E_R$ berechnete Wert weicht ebenfalls um etwa $1.54\sigma$ vom Literaturwert ab. Die Abweichung des von uns ermittelten Wertes von $\sigma_{13}$ zum Literaturwert von etwa $1.8$ beträgt ungefähr $1.13\sigma$.

Die bekannten Positionen der $K_{\alpha}$- und $K_{\beta}$-Linien im Röntgenfluoreszenzspektrum eines Metalls lassen sich außerdem verwenden, um die Bestandteile von Metalllegierungen herauszufinden. Um dies zu erproben, untersuchten wir im abschließenden Versuchsteil die Röntgenfluoreszenzspektren fünf verschiedener Metalllegierungen. Hierunter fanden wir die folgenden Legierungen:
\begin{itemize}
  \item Eisen + Chrom = Ferrochrom(?)
  \item Kupfer + Zink = Messing (zwei Mal)
  \item Eisen + Nickel = Invar(?)
  \item Kupfer + Gallium + Germanium = ?
\end{itemize}

Die aufgezeichneten Spektren sind noch einmal in den Abbildungen \abbref{fig:legierung1_zsmf} und \abbref{fig:legierung2345_zsmf} zusammengefasst zu sehen.


\begin{figure}[H]
  \centering
  \includegraphics[width=\textwidth]{files/legierung1.jpg}
  \caption{Legierung 1: Eisen + Chrom}
  \label{fig:legierung1_zsmf}
\end{figure}

\begin{figure}[H]
  \centering
  \includegraphics[width=\textwidth]{files/legierung2345.jpg}
  \caption{Legierungen 2 - 5}
  \label{fig:legierung2345_zsmf}
\end{figure}
	
	
\end{document}

