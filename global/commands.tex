% EINIGE COMMANDS

\newcommand{\R}[0]{\mathbb{R}}
\newcommand{\N}[0]{\mathbb{N}}
\newcommand{\C}[0]{\mathbb{C}}
\newcommand{\Z}[0]{\mathbb{Z}}
\newcommand{\Q}[0]{\mathbb{Q}}
\definecolor{lblue}{RGB}{51,153,255} % Farbe für Notizen und besondere Zeichen
\newcommand{\bbew}[0]{\bigg| \bigg|} % Um Endergebnisse doppel zu unterstreichen, außer dass die Striche nicht unten sondern zur Seite stehen
\newcommand{\baro}[0]{\textcolor{lblue}{| }} % Strich um Rechnung von Nebenrechnung/Kommentar zu trennen
\newcommand{\arrowroa}[0]{\textcolor{lblue}{\to}}
\newcommand{\ind}[0]{\ \ \ \ \ }
\newcommand{\indd}[0]{\ind \ind }
\newcommand{\evalint}[3]{\bqty{#1}_{#2}^{#3}} % Stammfunktion mit eckigen Klammern die automatisch auf die entsprechende Größe gesetzt werden
\newcommand{\diff}{\mathrm{\ d}} % d aber nicht in kursiv
\newcommand{\EQ}[0]{\mathrel{\stackon[1pt]{$=$}{$\scriptstyle!$}}} % Gleichheitszeichen mit ! oben drauf
\newcommand{\limit}[2]{\lim\limits_{#1 \to #2}}
\DeclareMathOperator{\Lagr}{\mathcal{L}} % Lagrnaginag
\newcommand{\ELG}[1]{\dv{}{t} \pdv{\Lagr}{\dot{#1}} - \pdv{\Lagr}{#1}} % ELG aber man muss nicht so viel schreiben
\def\tbs{\textbackslash} % backslash but shorter
\newcommand{\e}[1]{\text{e}^{#1}}
\newcommand{\opnorm}[1]{\norm{#1}_{\text{op}}}
\newcommand{\summ}[2]{\sum\limits_{#1}^{#2}}

\newcommand{\mean}[1]{\langle #1 \rangle}

\newcommand{\todo}[1]{\textcolor{red}{TODO: #1}}

\newcommand{\abbref}[1]{Abbildung (\ref{#1})}
\newcommand{\tabref}[1]{Tabelle (\ref{#1})}

\newcommand{\isot}[2]{$^{#1}\mathrm{#2}$}

% EINHEITEN

\newcommand{\gm}[0]{\text{ gm}}
\newcommand{\cm}[0]{\text{\,cm}}
\newcommand{\cmm}[0]{\text{ cm}}
\newcommand{\dm}[0]{\text{ dm}}
\newcommand{\km}[0]{\text{ km}}
\newcommand{\kg}[0]{\text{ kg}}
\newcommand{\m}[0]{\text{ m}}
\newcommand{\s}[0]{\text{ s}}
\newcommand{\J}[0]{\text{ J}}
\newcommand{\Newt}[0]{\text{ N}}
\newcommand{\rad}[0]{\text{ rad}}
\newcommand{\mm}[0]{\text{ mm}}
\newcommand{\mbar}[0]{\text{ mbar}}
\newcommand{\barp}[0]{\text{ bar}}
\newcommand{\mPa}[0]{\text{ mPa}}
\newcommand{\Pa}[0]{\text{ Pa}}
\newcommand{\K}[0]{\text{ K}}
\renewcommand{\l}[0]{\text{ l}}
\newcommand{\mol}[0]{\text{ mol}}
\newcommand{\Mol}[0]{\text{Mol}}
\newcommand{\V}[0]{\text{ V}}
\newcommand{\A}[0]{\text{ A}}
\newcommand{\mA}[0]{\text{ mA}}
\newcommand{\muA}[0]{\ \mu\text{A}}
\newcommand{\Ohm}[0]{\ \Omega}
\newcommand{\kOhm}[0]{\ \text{k}\Omega}
\newcommand{\mOhm}[0]{\ \text{m}\Omega}
\newcommand{\Celsius}[0]{\text{ $^\circ$C}}
\newcommand{\const}[0]{\text{ const.}}

\newcommand{\idf}[0]{\mathrm{id}}

\newcommand{\err}[1]{\Delta #1}
\newcommand{\errpdvsq}[2]{\qty(\pdv{#1}{#2} \cdot \err{#2})^2}
\newcommand{\errsq}[2]{\qty(#1 \cdot \err{#2})^2}
\newcommand{\errsqval}[2]{\qty(#1 \cdot #2)^2}
\newcommand{\errpdv}[2]{\pdv{#1}{#2} \cdot \err{#2}}

\newcommand{\chisq}{\chi^{2}}
\newcommand{\chisqrd}{\chi^{2}_{\mathrm{red}}}

%LA Zeug
\newcommand{\EndVS}[0]{\mathrm{End}} % Endomorphismen
\newcommand{\EigVS}[0]{\mathrm{Eig}} % Eigenraum
\newcommand{\IsomVS}[0]{\mathrm{Isom}} %Isometrien
\newcommand{\Ann}[0]{\mathrm{Ann}} %Annullator
\newcommand{\LinSpan}[0]{\mathrm{Span}} %Linearer Spann
\newcommand{\GL}[0]{\mathrm{GL}} % General Linear Group
\newcommand{\Mat}[3]{\mathrm{Mat}(#1,#2;#3)} % Matritzenraum
\newcommand{\idmat}[0]{\mathds{1}} % einheitsmatrix symbol
\newcommand{\idmatd}[2]{\mathds{1}_{#1 \times #2}} % einheitsmatrix symbol mit suffix
\newcommand{\im}[0]{\mathrm{im}} % Bildraum
\newcommand{\id}[0]{\mathrm{id}} % Identitätsabbildung
\newcommand{\modsp}[2]{{\raisebox{.2em}{$#1$}\left/\raisebox{-.3em}{$#2$}\right.}} % Faktor/Quotient
\newcommand{\longto}[0]{\longrightarrow} %langes äquivalent zu \to
\newcommand{\Hau}[2]{\mathrm{Hau_{#1}\qty(#2)}}

\newcommand{\zz}[0]{Z\kern-.3em\raise-0.5ex\hbox{Z}: } % zu zeigen / z.z.

\newcommand{\ffrac}[2]{\flatfrac{#1}{#2}}
\newcommand{\Thalf}[0]{T_{\flatfrac{1}{2}}}


% MATHEMATISCHE OPERATOREN DIE NICHT IN DEN PACKAGES SIND

\DeclareMathOperator{\arcosh}{arcosh}
\DeclareMathOperator{\acsinh}{arsinh}
\DeclareMathOperator{\artanh}{artanh}
\DeclareMathOperator{\arcsech}{arcsech}
\DeclareMathOperator{\arccsch}{arcCsch}
\DeclareMathOperator{\arccoth}{arcCoth}
